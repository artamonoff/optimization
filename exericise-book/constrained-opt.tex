% !TEX root = exercise-book-opt.tex

\begin{exercise}
Решите задачи оптимизации
\begin{align*}
	&\begin{gathered}
		\max(2x+3y) \\ s.t.\; 2x^2+y^2=11 
	\end{gathered} &
	&\begin{gathered}
		\min(2x+3y) \\ s.t.\; 2x^2+y^2=11 
	\end{gathered} \\
	&\begin{gathered}
		\max(5x-3y) \\ s.t.\; x^2+3y^2=28 
	\end{gathered} &
	&\begin{gathered}
		\min(5x-3y) \\ s.t.\; x^2+3y^2=28 
	\end{gathered}
\end{align*}
\end{exercise}

\begin{exercise}
Решите задачи оптимизации
\begin{align*}
	& \begin{gathered}
		\min(x^2+2y^2) \\ s.t.\;3x+2y=22
	\end{gathered} &
	& \begin{gathered}
		\max(10-2x^2-18y^2) \\ s.t.\;4x+6y=30
	\end{gathered} \\
	& \begin{gathered}
		\max(y^2-2x^2) \\ s.t.\;4x+3y=5
	\end{gathered} &
	& \begin{gathered}
		\min(2y^2-x^2) \\ s.t.\;5x+4y=17
	\end{gathered}
\end{align*}
\end{exercise}

\begin{exercise}
Решите задачи оптимизации
\begin{align*}
	& \begin{gathered}
		\max(x^2y^2) \\ s.t.\;3x+2y=24
	\end{gathered} &
	& \begin{gathered}
		\min(x^2y^2) \\ s.t.\;3x+2y=24
	\end{gathered}
\end{align*}
\end{exercise}

\begin{exercise}
Решите задачи оптимизации
\begin{align*}
	& \begin{gathered}
		\max(xy) \\ s.t.\;x^2+2y^2=36
	\end{gathered} &
	& \begin{gathered}
		\min(xy) \\ s.t.\;x^2+2y^2=36
	\end{gathered}
\end{align*}
\end{exercise}

\begin{exercise}
Найти экстремум функции полезности $u=x^2y$ при бюджетном ограничении
$2x+3y=90$. 
% Дайте экономическую интерпретацию параметров функции полезности.
\end{exercise}

\begin{exercise}
Для производства предприятие закупает два вида ресурсов по ценам
$P_1=10$ и $P_2=20$, бюджет составляет \$1200. Производственная
функция предприятия равна $f(x,y)=\sqrt{xy}$. 
Найдите количество ресурсов с целью обеспечения оптимальной производственной программы.
% Дайте экономическую интерпретацию производственной функции и ее параметров.
\end{exercise}

\begin{exercise}
Для производства предприятие закупает два вида ресурсов по ценам
$P_x=5$ и $P_y=2$, бюджет составляет \$200. Производственная
функция предприятия равна $f(x,y)=2\sqrt{xy}$.
Найдите количество ресурсов с целью обеспечения оптимальной производственной программы.
% \begin{enumerate}
% 	\item Какой экономической ситуации соответствует экзогенность цен?
% 	\item  Будет ли производственная функция однородной? Если да, то какой степени и дайте
% 	интерпретацию степени однородности.
% 	\item Постройте модель для нахождения оптимального производства. %производственной программы.
% 	\item Приведите необходимые условия экстремума.
% 	\item Приведите достаточные условия экстремума.
% 	\item Какое количество ресурсов необходимо закупить?
% 	\item Дайте экономическую интерпретацию множителя Лагранжа.
% \end{enumerate}
\end{exercise}

\begin{exercise}
Производственная функция предприятия равна $f(x,y)=\sqrt{xy}$. Ресурсы
закупаются по ценам $P_1$ и $P_2$. Рассмотрим задачу оптимизации
\begin{gather*}
	\min (P_1x+P_2 y) \\  f(x,y)=Q_0
\end{gather*}
Дайте интерпретацию оптимальной задачи и найдите её решение.
% \begin{enumerate}
% 	\item Дайте интерпретацию экстремальной задачи с ограничениями
% 	\item Напишите функцию Лагранжа и необходимые условия
% 	экстремума.
% 	\item Сформулируйте достаточные условия экстремума.
% 	\item Найдите решения экстремальной задачи.
% 	\item Дайте экономическую интерпретацию множителя Лагранжа.
% 	Как (экономически) можно объяснить, что множитель Лагранжа не зависит то
% 	объема выпуска $Q_0$?
% \end{enumerate}  
\end{exercise}

\begin{exercise}
Потребительская корзина состоит их трех товаров, цена на которые равны
$P_1$, $P_2$, $P_3$. Доход равен $I$.  Функция полезности потребителя равна
\[
	U(q_1,q_2,q_3)=\ln q_1+\ln q_2+\ln q_3.
\]
Найдите оптимальную потребительскую корзину.
% \begin{enumerate}
% 	\item Постройте модель оптимизации для нахождения оптимальной
% 	потребительской корзины.
% 	\item Сформулируйте необходимые и достаточные условия экстремума.
% 	\item Найдите оптимальную потребительскую корзину.
% 	\item Дайте экономическую интерпретацию множителя Лагранжа.
% \end{enumerate}
\end{exercise}

\begin{exercise}
В условиях предыдущей задачи рассмотрите функцию полезности
\begin{align*}
	U(q_1,q_2,q_3)&=a\ln q_1+b\ln q_2+c\ln q_3 &
	a,b,c&>0
\end{align*}
\end{exercise}

\begin{exercise}
Фирма для производства использует два фактора производства: капитал и труд.
Производственная функция имеет вид $F=3KL^2$. Фирма решает следующую задачу
\begin{gather*}
	\min(5K+4L) \\ F(K,L)=9600
\end{gather*}
Дайте интерпретацию оптимальной задачи и найдите её решение. 
% Дайте интерпретацию множителям Лагранжа.
\end{exercise}

\begin{exercise}
Решите задачи оптимизации
\begin{gather*}
	\min(2x^2+2y^2+4z^2+2xy+2xz+2yz-10x-50y-10z) \\ 
	s.t.\;x+2y+3z=20 \\
	\max(10-9x-3y+3z-4x^2-2y^2-2z^2+4xy+2xz+2yz) \\
	s.t.\; 2x+2y-4z=7
\end{gather*}
\end{exercise}

\begin{exercise}
Решите \textbf{численно}\footnote{MS Excel/Python} задачи оптимизации
\begin{align*}
	&\max (x+y+z) & &\min(x^2+y^2+z^2) \\
	s.t.&\left\{\begin{aligned}
		2x^2+y^2+z^2 &= 9 \\ x-y+z&=0 
	\end{aligned}\right. &
	s.t.&\left\{\begin{aligned}
		2x+y+2z &= 10 \\ 3x-2y+z&=6 
	\end{aligned}\right. \\
	&\min (x+y+z) & &\min(x^2+y^2+z^2) \\
	s.t.&\left\{\begin{aligned}
		x^2+2y^2+2z^2 &= 16 \\ x+y-z&=0 
	\end{aligned}\right. &
	s.t.&\left\{\begin{aligned}
		2x^2+4y^2+3z^2 &= 16 \\ x+y-z&=0 
	\end{aligned}\right.
\end{align*}
\end{exercise}