\documentclass[12pt]{article}

\usepackage[utf8]{inputenc}
\usepackage[T2A]{fontenc}
\usepackage[english, russian]{babel}
\usepackage{amsmath, amsthm, amssymb}
\usepackage{enumerate,hhline, bm}
\usepackage[mathscr]{eucal}
%\usepackage{mathtext}

\usepackage{hyperref}
\hypersetup{unicode=true, final=true, colorlinks=true}



%
%  Линейная алгебра
%
\DeclareMathOperator{\rank}{rank}
\DeclareMathOperator{\dimension}{dim}
\newcommand{\LinearSpace}{{\mathfrak L}}
\DeclareMathOperator{\trace}{tr} % След матрицы
\newcommand{\Lin}{{\mathfrak H}}

%
%   Числовые
%
\newcommand{\Complex}{{\mathbb C}}
\newcommand{\N}{\mathbb N}
\newcommand{\Z}{{\mathbb Z}}
\newcommand{\Q}{{\mathbb Q}}
\newcommand{\R}{{\mathbb R}}
\newcommand{\semiaxes}{{\mathbb R_+}}
\DeclareMathOperator{\Real}{Re}
\DeclareMathOperator{\Image}{Im}
\DeclareMathOperator{\dist}{dist}

%
%   Вектора
%
\newcommand{\vconst}{{\mathbf const}}
\newcommand{\vectx}{{\bm x}}
\newcommand{\vecty}{{\bm y}}
\newcommand{\vectz}{{\bm z}}
\newcommand{\vecte}{{\bm e}}
\newcommand{\vectw}{{\bm w}}
\newcommand{\vecth}{{\bm h}}
\newcommand{\vectr}{{\bm r}}
\newcommand{\vectq}{{\bm q}}
\newcommand{\vectf}{{\bm f}}%{\boldsymbol{f}}
\newcommand{\vectg}{{\bm g}}%{\boldsymbol{f}}
\newcommand{\vectu}{{\bm u}}
\newcommand{\vectv}{{\bm v}}
\newcommand{\vectc}{{\bm c}}
\newcommand{\vectalpha}{{\bm{\alpha}}}
\newcommand{\vectbeta}{{\bm{\beta}}}
\newcommand{\vectgamma}{{\bm{\gamma}}}
\newcommand{\vectdelta}{{\bm{\delta}}}
\newcommand{\vecteta}{{\bm{\eta}}}
\newcommand{\vectpi}{{\bm{\pi}}}
\newcommand{\vectmu}{{\bm{\mu}}}
\newcommand{\vectlambda}{{\bm{ \lambda}}}

% 
%  Матрицы
%
\newcommand{\Id}{I}
\newcommand{\matrixX}{{\bm X}}
\newcommand{\matrixY}{{\bm Y}}
\newcommand{\matrixU}{{\bm U}}
\newcommand{\matrixV}{{\bm V}}
\newcommand{\matrixR}{{\bm R}}
\newcommand{\matrixZ}{{\bm Z}}
\newcommand{\matrixA}{{\bm A}}
\newcommand{\matrixB}{{\bm B}}
\newcommand{\matrixQ}{{\bm Q}}
\newcommand{\matrixH}{{\bm H}}
\newcommand{\matrixGamma}{{\bm{\Gamma}}}
\newcommand{\matrixPi}{{\bm{\Pi}}}
\newcommand{\Minor}{{\mathcal M}} % Минор матрицы.

%
%  Мат.анализ
%
\newcommand{\Hamiltonian}{{\mathscr H}} % Гамильтониан
\newcommand{\HamiltonianG}{{\mathfrak H}} % Гамильтониан
\newcommand{\Hessian}{{\mathsf {Hess}}} % Hessian matrix
\newcommand{\BordHessian}{\mathsf{Hess}} % Bordered Hessian
\newcommand{\DomV}{{\mathcal V}}
\newcommand{\DomU}{{\mathcal U}}
\newcommand{\DomD}{{\mathscr D}}
\newcommand{\FuncF}{{\mathscr F}} % Функционал F
\newcommand{\FuncL}{{\mathscr L}} % Функционал L
\newcommand{\Lagrange}{{\mathscr L}}
\newcommand{\LagrangeZ}{{\mathscr Z}}
\DeclareMathOperator{\Domain}{dom} %Область определения
\DeclareMathOperator{\argmax}{argmax}
\DeclareMathOperator{\argmin}{argmin}
\DeclareMathOperator{\grad}{grad}

%
% Теоремы, Примеры etc
%
\theoremstyle{plain}
\newtheorem*{teorema}{Теорема}
\newtheorem*{importante}{Важно!}
\newtheorem*{ejemplo}{Пример}
\newtheorem*{definicion}{Определение}
\newtheorem*{col}{Следствие}
\newtheorem*{propuesta}{Предложение}

\theoremstyle{remark}
\newtheorem*{remark}{Замечание}

\theoremstyle{remark}
\newtheorem{exercise}{}[subsection]
\renewcommand{\theexercise}{\textbf{\textnumero \arabic{exercise}}}

%\DeclareMathOperator{\cov}{cov}
%\DeclareMathOperator{\corr}{corr}
%\DeclareMathOperator{\Var}{Var}

%\topmargin=-2cm%-1.5cm

%\addtolength{\textheight}{3cm}

%\oddsidemargin=-0.1cm

%\addtolength{\textwidth}{1.8cm}


\title{Задачи по Методам оптимизации и Теории игр}
\author{Артамонов Н.В.}
%\date{весна 2014}

%\title{Задачи для подготовки к экзамену по курсу 
%<<Методы оптимальных решений>>}\author{\copyright Артамонов Н.В., кафедра ЭММАЭ}

\begin{document}

\maketitle

%\markright{}
\tableofcontents

\section{Задачи оптимизации}

\textbf{Внимание: Во всех расчетных задачах обязательно проверять достаточные условия экстремума!
Задачи, отмеченные звёздочками, не является обязательными}

\subsection{Безусловная оптимизация}

% !TEX root = exercise-book-opt.tex

\begin{exercise}
Найдите локальные экстремумы функций
\begin{align*}
	f(x,y) &= 10-6x-4y+2x^2+y^2-2xy \\
	f(x,y) &= 8+8x+4y-5x^2-2y^2+6xy \\
	f(x,y) &= 5+2x+6y+5x^2+3y^2+8xy
\end{align*}
% Нарисуйте графики функций (MS Excel, Python etc)
\end{exercise}

\begin{exercise}
Найдите локальные экстремумы функций
\begin{align*}
	f(x,y,z) &= 6+4x+2y+6z+2x^2+2y^2+z^2+2xy+2yz \\
	f(x,y,z) &= 3+4x+8y+4z-3x^2-2y^2-4z^2+2xy+2xz+4yz\\
	f(x,y,z) &= 8+2x+4y+2z+2x^2+y^2+3z^2+2xy+4xz+4yz
\end{align*}
\end{exercise}

\begin{exercise}
Найдите локальные экстремумы функций
\begin{align*}
	f(x,y) &= 5+x^3-y^3+3xy \\
	f(x,y) &= 3x^2y+y^3-3x^2-3y^2+2 \\
	f(x,y) &= x^3+x^2y-2y^3+6y
\end{align*}
\end{exercise}

\begin{exercise}
Завод производит три вида товаров и продает их по ценам $P_1=2$, $P_2=4$ и $P_3=6$.
Издержки производства равны
\[
	C(Q_1,Q_2,Q_3)=2Q_1^2+Q_2^2+2Q_3^2-2Q_2Q_3
\]
($Q_1, Q_2, Q_3$ -- объемы производства товаров). Найдите оптимальные объемы производства. 
% Какой экономической ситуации соответствует экзогенность цен?
\end{exercise}

\begin{exercise}
Завод производит три вида товаров и продает их по ценам $P_1=2$, $P_2=2$ и $P_3=3$.
Издержки производства равны 
\[
	C(Q_1,Q_2,Q_3)=2Q_1^2+Q_2^2+2Q_3^2-2Q_2Q_3-2Q1Q_3
\]
($Q_1, Q_2, Q_3$ -- объемы производства товаров).
Найдите оптимальные объемы производства.
\end{exercise}

\begin{exercise}
Завод производит два вида товаров, (обратные) функции спроса на которые
имеют вид $P_1=21-5Q_1+2Q_2$ и $P_2=35-Q_2+2Q_1$. Функция
издержек равна
\[
	C(Q_1,Q_2)=Q_1+3Q_2
\]
($Q_1, Q_2$ -- объемы производства товаров). Найдите оптимальные объемы производства. 
% Какой экономической ситуации соответствует эндогенность цен?
\end{exercise}

\begin{exercise}
Завод производит два вида товаров, (обратные) функции спроса на которые
имеют вид $P_1=51-2Q_1+3Q_2$ и $P_2=47-5Q_2+3Q_1$. Функция
издержек равна
\[
	C(Q_1,Q_2)=3Q_1+5Q_2
\]
($Q_1, Q_2$ -- объемы производства товаров). Найдите оптимальные объемы производства. 
% Какой экономической ситуации соответствует эндогенность цен?
\end{exercise}

\begin{exercise}
Найдите локальные экстремумы функций
\begin{align*}
	f(x,y) &= 6\ln x+8\ln y-3x-2y \\
	f(x,y) &= 4\ln x+6\ln y+2x-3xy \\
	f(x,y) &= 5\ln x+4\ln y-x-4xy
\end{align*}
\end{exercise}

\subsection{Выпуклость}

% !TEX root = exercise-book-opt.tex

\begin{exercise}
Исследуйте на выпуклость/вогнутость функции на \(\R^2\)
\begin{align*}
	f(x,y) &= 10-6x-4y+2x^2+y^2-2xy \\
	f(x,y) &= 8+8x+4y-5x^2-2y^2+6xy \\
	f(x,y) &= 5+2x+6y+5x^2+3y^2+8xy \\
	f(x,y) &= 10+x^2+y^2+2xy \\
	f(x,y) &= 5+4xy-2x^2-2y^2
\end{align*}
\end{exercise}

\begin{exercise}
Исследуйте на выпуклость/вогнутость функции на \(\R^3\)
\begin{align*}
	f(x,y,z) &= 6+4x+2y+6z+2x^2+2y^2+z^2+2xy+2yz \\
	f(x,y,z) &= 3+4x+8y+4z-3x^2-2y^2-4z^2+2xy+2xz+4yz\\
	f(x,y,z) &= 8+2x+4y+2z+2x^2+y^2+3z^2+2xy+4xz+4yz
\end{align*}
\end{exercise}

\begin{exercise}
При каких значениях параметра $\beta$ функция
\[
	f(x_1,x_2,x_3)=2x_1^2+4x_2^2+x_3^2-\beta x_1x_3
\]
будет строго/нестрого выпуклой? Строго/нестрого вогнутой?
\end{exercise}

\begin{exercise}
Исследуйте на выпуклость/вогнутость функции, определённые на 
\(\R^2_+\) (\(a,b>0\))
\begin{align*}
	f(x,y) &= a\ln x+b\ln y-2x^2-y^2-2xy \\
	f(x,y) &= x^2+5y^2-4xy-a\ln x-b\ln y \\
	f(x,y) &= a\ln x+b\ln y -3x^2-5y^2-8xy
\end{align*}
\end{exercise}

\subsection{Оптимизации с ограничениями  равенства}

% !TEX root = exercise-book-opt.tex

\begin{exercise}
Решите задачи оптимизации
\begin{align*}
	&\begin{gathered}
		\max(2x+3y) \\ s.t.\; 2x^2+y^2=11 
	\end{gathered} &
	&\begin{gathered}
		\min(2x+3y) \\ s.t.\; 2x^2+y^2=11 
	\end{gathered} \\
	&\begin{gathered}
		\max(5x-3y) \\ s.t.\; x^2+3y^2=28 
	\end{gathered} &
	&\begin{gathered}
		\min(5x-3y) \\ s.t.\; x^2+3y^2=28 
	\end{gathered}
\end{align*}
\end{exercise}

\begin{exercise}
Решите задачи оптимизации
\begin{align*}
	& \begin{gathered}
		\min(x^2+2y^2) \\ s.t.\;3x+2y=22
	\end{gathered} &
	& \begin{gathered}
		\max(10-2x^2-18y^2) \\ s.t.\;4x+6y=30
	\end{gathered} \\
	& \begin{gathered}
		\max(y^2-2x^2) \\ s.t.\;4x+3y=5
	\end{gathered} &
	& \begin{gathered}
		\min(2y^2-x^2) \\ s.t.\;5x+4y=17
	\end{gathered}
\end{align*}
\end{exercise}

\begin{exercise}
Решите задачи оптимизации
\begin{align*}
	& \begin{gathered}
		\max(x^2y^2) \\ s.t.\;3x+2y=24
	\end{gathered} &
	& \begin{gathered}
		\min(x^2y^2) \\ s.t.\;3x+2y=24
	\end{gathered}
\end{align*}
\end{exercise}

\begin{exercise}
Решите задачи оптимизации
\begin{align*}
	& \begin{gathered}
		\max(xy) \\ s.t.\;x^2+2y^2=36
	\end{gathered} &
	& \begin{gathered}
		\min(xy) \\ s.t.\;x^2+2y^2=36
	\end{gathered}
\end{align*}
\end{exercise}

\begin{exercise}
Найти экстремум функции полезности $u=x^2y$ при бюджетном ограничении
$2x+3y=90$. 
% Дайте экономическую интерпретацию параметров функции полезности.
\end{exercise}

\begin{exercise}
Для производства предприятие закупает два вида ресурсов по ценам
$P_1=10$ и $P_2=20$, бюджет составляет \$1200. Производственная
функция предприятия равна $f(x,y)=\sqrt{xy}$. 
Найдите количество ресурсов с целью обеспечения оптимальной производственной программы.
% Дайте экономическую интерпретацию производственной функции и ее параметров.
\end{exercise}

\begin{exercise}
Для производства предприятие закупает два вида ресурсов по ценам
$P_x=5$ и $P_y=2$, бюджет составляет \$200. Производственная
функция предприятия равна $f(x,y)=2\sqrt{xy}$.
Найдите количество ресурсов с целью обеспечения оптимальной производственной программы.
% \begin{enumerate}
% 	\item Какой экономической ситуации соответствует экзогенность цен?
% 	\item  Будет ли производственная функция однородной? Если да, то какой степени и дайте
% 	интерпретацию степени однородности.
% 	\item Постройте модель для нахождения оптимального производства. %производственной программы.
% 	\item Приведите необходимые условия экстремума.
% 	\item Приведите достаточные условия экстремума.
% 	\item Какое количество ресурсов необходимо закупить?
% 	\item Дайте экономическую интерпретацию множителя Лагранжа.
% \end{enumerate}
\end{exercise}

\begin{exercise}
Производственная функция предприятия равна $f(x,y)=\sqrt{xy}$. Ресурсы
закупаются по ценам $P_1$ и $P_2$. Рассмотрим задачу оптимизации
\begin{gather*}
	\min (P_1x+P_2 y) \\  f(x,y)=Q_0
\end{gather*}
Дайте интерпретацию оптимальной задачи и найдите её решение.
% \begin{enumerate}
% 	\item Дайте интерпретацию экстремальной задачи с ограничениями
% 	\item Напишите функцию Лагранжа и необходимые условия
% 	экстремума.
% 	\item Сформулируйте достаточные условия экстремума.
% 	\item Найдите решения экстремальной задачи.
% 	\item Дайте экономическую интерпретацию множителя Лагранжа.
% 	Как (экономически) можно объяснить, что множитель Лагранжа не зависит то
% 	объема выпуска $Q_0$?
% \end{enumerate}  
\end{exercise}

\begin{exercise}
Потребительская корзина состоит их трех товаров, цена на которые равны
$P_1$, $P_2$, $P_3$. Доход равен $I$.  Функция полезности потребителя равна
\[
	U(q_1,q_2,q_3)=\ln q_1+\ln q_2+\ln q_3.
\]
Найдите оптимальную потребительскую корзину.
% \begin{enumerate}
% 	\item Постройте модель оптимизации для нахождения оптимальной
% 	потребительской корзины.
% 	\item Сформулируйте необходимые и достаточные условия экстремума.
% 	\item Найдите оптимальную потребительскую корзину.
% 	\item Дайте экономическую интерпретацию множителя Лагранжа.
% \end{enumerate}
\end{exercise}

\begin{exercise}
В условиях предыдущей задачи рассмотрите функцию полезности
\begin{align*}
	U(q_1,q_2,q_3)&=a\ln q_1+b\ln q_2+c\ln q_3 &
	a,b,c&>0
\end{align*}
\end{exercise}

\begin{exercise}
Фирма для производства использует два фактора производства: капитал и труд.
Производственная функция имеет вид $F=3KL^2$. Фирма решает следующую задачу
\begin{gather*}
	\min(5K+4L) \\ F(K,L)=9600
\end{gather*}
Дайте интерпретацию оптимальной задачи и найдите её решение. 
% Дайте интерпретацию множителям Лагранжа.
\end{exercise}

\begin{exercise}
Решите задачи оптимизации
\begin{gather*}
	\min(2x^2+2y^2+4z^2+2xy+2xz+2yz-10x-50y-10z) \\ 
	s.t.\;x+2y+3z=20 \\
	\max(10-9x-3y+3z-4x^2-2y^2-2z^2+4xy+2xz+2yz) \\
	s.t.\; 2x+2y-4z=7
\end{gather*}
\end{exercise}

\begin{exercise}
Решите \textbf{численно}\footnote{MS Excel/Python} задачи оптимизации
\begin{align*}
	&\max (x+y+z) & &\min(x^2+y^2+z^2) \\
	s.t.&\left\{\begin{aligned}
		2x^2+y^2+z^2 &= 9 \\ x-y+z&=0 
	\end{aligned}\right. &
	s.t.&\left\{\begin{aligned}
		2x+y+2z &= 10 \\ 3x-2y+z&=6 
	\end{aligned}\right. \\
	&\min (x+y+z) & &\min(x^2+y^2+z^2) \\
	s.t.&\left\{\begin{aligned}
		x^2+2y^2+2z^2 &= 16 \\ x+y-z&=0 
	\end{aligned}\right. &
	s.t.&\left\{\begin{aligned}
		2x^2+4y^2+3z^2 &= 16 \\ x+y-z&=0 
	\end{aligned}\right.
\end{align*}
\end{exercise}

\subsection{Оптимизация с ограничениями неравенства}

% !TEX root = exercise-book-opt.tex

\begin{exercise}
Решите задачи оптимизации
\begin{align*}
	&\begin{gathered}
		\min(2x^2+5y^2) \\  s.t\; x-y\geq 5
	\end{gathered} &
	&\begin{gathered}
		\max (x+2y) \\ s.t.\; x^2+y^2\leq4
	\end{gathered}
\end{align*}
\end{exercise}

\begin{exercise}
Решите задачи оптимизации
\begin{align*}
	&\begin{gathered}
		\max(2x+4y-5x^2-y^2-4xy)\\ s.t.\; 2x+3y\leq 40
	\end{gathered} &
	&\begin{gathered}
		\max(4x+8y-6x^2-2y^2-8xy)\\ s.t.\; 2x+2y\leq 9
	\end{gathered} \\
	&\begin{gathered}
		\max(4x+10y-2x^2-6y^2-6xy)\\ s.t.\; 3x+6y\leq 3
	\end{gathered} &
	&\begin{gathered}
		\max(8x+8y-3x^2-3y^2-6xy)\\ s.t.\; 2x+4y\leq 13
	\end{gathered}
\end{align*}
\end{exercise}

\begin{exercise}
Решите задачи оптимизации
\begin{align*}
	&\begin{gathered}
		\min(5x^2+y^2+4xy-2x-4y)\\ s.t.\; 2x+y\geq 40
	\end{gathered} &
	&\begin{gathered}
		\min(5x^2+y^2-4xy-2x-y)\\ s.t.\; 2x+3y\geq 10
	\end{gathered} \\
	&\begin{gathered}
		\min(5x^2+4y^2-8xy-3x-4y)\\ s.t.\; 2x+4y\geq 31
	\end{gathered} &
	&\begin{gathered}
		\min(2x^2+6y^2-6xy-4x-43)\\ s.t.\; 3x+6y\geq 20
	\end{gathered}
\end{align*}
\end{exercise}

\begin{exercise}
Решите задачи оптимизации
\begin{align*}
	& \begin{gathered}
		\max (x-2y) \\ s.t.\; x^2+y^2\leq 4
	\end{gathered} &
	& \begin{gathered}
		\max(1-(x+1)^2-(y-1)^2) \\ s.t.\; x+y\leq 10
	\end{gathered} \\
	& \begin{gathered}
		\max (x-2y) \\ s.t.\left\{\begin{aligned}
			 x^2+y^2&\leq 4 \\  x,y&\geq0
		\end{aligned}\right.
	\end{gathered} &
	& \begin{gathered}
		\max(1-(x+1)^2-(y-1)^2) \\ s.t.\left\{\begin{aligned}
			x+y&\leq 10  \\  x,y&\geq0
		\end{aligned}\right.
	\end{gathered}
\end{align*}
\end{exercise}

% \begin{exercise}
% Завод производит два вида товаров, цена на которые равны 
% $P_1=50$ и $P_2=40$. Функция издержек равна
% \[
% 	C(Q_1,Q_2)=2Q_1^2+Q_2^2
% \]
% ($Q_1, Q_2$ -- объемы производства товаров). Найдите оптимальные объемы производства,
% максимизирующие выручку, если издержки не должны
% превышать 20. 
% \end{exercise}

% \begin{exercise}%[\textbf{5 баллов} за каждый пункт]
% Завод производит два вида товаров, (обратные) функции спроса на которые
% имеют вид $P_1=50-2Q_1$ и $P_2=40-2Q_2$ (цены эндогенны). Функция
% издержек равна $C(Q_1,Q_2)=Q_1+Q_2$
% ($Q_1, Q_2$ -- объемы производства товаров).  Производитель определил,
% что издержки не должны превышать 100. Найдите оптимальную производственную программу.
% \end{exercise}

% \begin{exercise}%[\textbf{5 баллов} за каждый пункт]
% Завод производит два вида товаров, цена на которые
% равны $P_1=2$ и $P_2=4$ (цены экзогенны). Функция
% издержек равна $C(Q_1,Q_2)=Q_1^2+2Q_2^2$
% ($Q_1, Q_2$ -- объемы производства товаров). Производитель определил, что
% выручка не должна быть меньше 20.
% Найдите оптимальную производственную программу.
% \end{exercise}

% \begin{exercise}
% %\textcolor{red}{Сложная задача. }
% Потребительская корзина состоит из двух товаров, её функция
% полезности равна $U(x,y)=x+a\ln(y)$ (параметр $a>0$).
% Потребитель решает оптимальную задачу
% \begin{align*}
% 	& \max\, U(x,y) \\ 
% 	s.t.&\left\{\begin{aligned}
% 		2x+y&\leq 10 \\ x,y&\geq0
% 	\end{aligned}\right.
% \end{align*}
% При каких значениях параметрах $a$ 
% \begin{enumerate}
% 	\item потребительская корзина состоит только из 
% 	второго товара
% 	\item содержит оба товара
% \end{enumerate}
% \end{exercise}

% \begin{exercise}[Consumption--Leisure choice]
% Экономический агент имеет два <<товара>>: <<отдых>> $l$ (leisure, в часах) и потребление $x$.
% Пусть $w$ -- почасовая оплата и $P$ -- цена потребления. Агент располагает общим временем $H$,
% которое он может тратить на работу и на отдых, и также имеет фиксированный доход $M$
% (non-labor income). Функция полезности экономического агента $U(x,l)x+c\ln l$ ($c>0$). 
% Рассмотрим задачу оптимизации
% \begin{gather*}
% 	\max U(x,l) \\
% 	s.t.\left\{\begin{gathered}
% 		Px+wl\leq wH+M \\ 0\leq l\leq H \\ x\geq0
% 	\end{gathered}
% 	\right.
% \end{gather*}
% Найдите решение задачи оптимизации.
% \end{exercise}

% \begin{exercise}
% Экономический агент потребляет два товара и его функция полезности равна
% $U(x,y)=y+c\ln x$ ( $c>0$). Цены на товары равны $P_1$ и $P_2$, доход равен $I$.
% Сформулируйте задачу об оптимальной потребительской корзине и найдите её решение.
% \end{exercise}

\subsection{Линейное программирование}

% !TEX root = exercise-book-opt.tex

\begin{exercise}
Рассмотрим задачи линейного программирования
\begin{align*}
	& \begin{gathered}
		\max(3x+5y) \\
		s.t.\left\{\begin{aligned}
			x+y &\leqslant5 \\ 2x+y &\leqslant8 \\ x,y&\geqslant0
		\end{aligned}\right.
	\end{gathered} &
	& \begin{gathered}
		\max(7x+4y) \\
		s.t.\left\{\begin{aligned}
			2x+5y &\leqslant30 \\ 2x+y &\leqslant14 \\ x,y&\geqslant0
		\end{aligned}\right.
	\end{gathered} \\
	& \begin{gathered}
		\max(4x+5y) \\
		s.t.\left\{\begin{aligned}
			x+3y &\leqslant15 \\ 4x+3y &\leqslant24 \\ x,y&\geqslant0
		\end{aligned}\right.
	\end{gathered} &
	& \begin{gathered}
		\max(8x+3y) \\
		s.t.\left\{\begin{aligned}
			2x+5y &\leqslant35 \\ 5x+3y &\leqslant40 \\ x,y&\geqslant0
		\end{aligned}\right.
	\end{gathered}
\end{align*}
\begin{enumerate} 
	\item Решите графическии (прямую) задачу
	\item Напишите и решите графически двойственную задачу.
\end{enumerate}
\end{exercise}

\begin{exercise}
Решите графически следующие задачи оптимизации
\begin{align*}
	& \begin{gathered}
		\max(5x+4y) \\
		s.t.\left\{\begin{aligned}
			x+3y &\leqslant18 \\ x+2y &\leqslant13 \\ 
			3x+2y &\leqslant27 \\ x,y&\geqslant0
		\end{aligned}\right.
	\end{gathered} &
	& \begin{gathered}
		\max(4x+3y) \\
		s.t.\left\{\begin{aligned}
			x+4y &\leqslant 28 \\ 2x+3y &\leqslant 26 \\ 
			x+y &\leqslant 11 \\ 2x+y &\leqslant 20 \\ x,y&\geqslant0
		\end{aligned}\right.
	\end{gathered} \\
	& \begin{gathered}
		\min(7x+6y) \\
		s.t.\left\{\begin{aligned}
			3x+y &\geqslant 9 \\ 4x+3y &\geqslant 22 \\ 
			x+3y &\leqslant 10 \\ x,y&\geqslant0
		\end{aligned}\right.
	\end{gathered} &
	& \begin{gathered}
		\min(2x+5y) \\
		s.t.\left\{\begin{aligned}
			3x+y &\geqslant 10 \\ 2x+y &\geqslant 8 \\ 
			x+3y &\geqslant 9 \\ x+6y &\geqslant 12 \\ x,y&\geqslant0
		\end{aligned}\right.
	\end{gathered}
\end{align*}
\end{exercise}

\begin{exercise}
Решите графически следующие задачи оптимизации
\begin{align*}
	& \begin{gathered}
		\max(5x-4y) \\
		s.t.\left\{\begin{aligned}
			2x+5y &\leqslant 16 \\ x+2y &\leqslant 22 \\ 
			-5x-y &\leqslant 19 \\ -2x+6y &\leqslant 14
		\end{aligned}\right.
	\end{gathered} &
	& \begin{gathered}
		\max(-3x+2y) \\
		s.t.\left\{\begin{aligned}
			x+2y &\leqslant 9 \\ 3x-y &\leqslant 13 \\ 
			-2x-7y &\leqslant 22 \\ -5x+y &\leqslant 18 \\ 
			-x+4y &\leqslant 15
		\end{aligned}\right.
	\end{gathered} 
\end{align*}
\end{exercise}

% \begin{exercise}
% Рассмотрим задачу линейного программирование в матричном виде 
% \begin{gather*}
% 	\max(\vectf^\top\vectx) \\
% 	s.t.\left\{\begin{aligned}
% 		\matrixA\vectx & \leq\vectc \\ \vectx&\geq0
% 	\end{aligned}\right.
% \end{gather*}
% Для каждого из примеров решите 
% \end{exercise}

% \begin{exercise}
% Найдите решение задачи оптимизации
% \begin{gather*}
% 	\min (3x_1+4x_2) \\ 
% 	\left\{\begin{aligned} 
% 	& 3x_1+2x_2\geq 13 \\ & 5x_1+x_2\geq 10 \\ 
% 	& x_1+2x_2\geq 7 \\ & x_1,x_2\geq0
% 	\end{aligned}\right.
% \end{gather*}
% \end{exercise}

% \begin{exercise}
% Решите задачу оптимизации
% \begin{gather*}
% 	\max (3x_1+4x_2+2x_3+x_4)  \\
% 	\left\{\begin{aligned}
% 	2x_1+x_2+5x_3+5x_4 &\leq  40 \\
% 	x_1+2x_2+3x_3+2x_4 &\leq 30 \\
% 	x_1,x_2,x_3,x_4 & \geq 0
% 	\end{aligned}\right.
% \end{gather*}
% \begin{enumerate}
% 	\item симплекс-методом
% 	\item через двойственную задачу
% \end{enumerate}
% \end{exercise}

% \begin{exercise}
% Решите задачу оптимизации
% \begin{gather*}
% 	\max(x_1+2x_2+2x_3)\\
% 	\left\{\begin{aligned}
% 	   3x_1+x_2+x_3 &\leq 20 \\
% 		x_1+2x_2+2x_3 &\leq30 \\
% 		x_1, x_2, x_3 &\geq0
% 	\end{aligned}\right.
% \end{gather*}
% \begin{enumerate}
% 	\item симплекс-методом
% 	\item через двойственную задачу
% \end{enumerate}
% \end{exercise}

% \begin{exercise}
% Решите задачу оптимизации с использованием симплекс-метода
% \begin{gather*}
% 	\max(x_1+2x_2+2x_3+4x_4)\\
% 	\left\{\begin{aligned}
% 	   2x_1+2x_2+x_3+4x_4 &\leq 30 \\ 
% 		x_1+2x_2+2x_3+2x_4 &\leq 50 \\ 
% 		x_1, x_2, x_3, x_4 &\geq0
% 	 \end{aligned}\right.
% \end{gather*}
% \begin{enumerate}
% 	\item симплекс-методом
% 	\item через двойственную задачу
% \end{enumerate}
% \end{exercise}

% \begin{exercise}
% Решите задачу оптимизации
% \begin{gather*}
% 	\max(2x_1+2x_2+x_3+x_4)\\
% 	\left\{\begin{aligned}
% 		2x_1+2x_2+x_3+x_4 &\leq 20 \\ 
% 		x_1+x_2+2x_3+2x_4 &\leq 30 \\ 
% 		x_1, x_2, x_3, x_4 &\geq0
% 	\end{aligned}\right.
% \end{gather*}
% \begin{enumerate}
% 	\item симплекс-методом
% 	\item через двойственную задачу
% \end{enumerate}
% \end{exercise}

\begin{exercise}
Фирма производит четыре товара и использует для производства два ресурса.
Норма затрат ресурсов, количество ресурсов и прибыль от каждой единицы
товара приведены в таблице
\begin{center}
\begin{tabular}{|c|c|c|c|c||c|}
	\hline
	& Товар 1 & Товар 2 & Товар 3 & Товар 4 &  Количество \\
	& & & & &  ресурса \\
	\hline
	Ресурс 1 & 4 & 4 & 1 & 2 & 100\\ \hline
	Ресурс 2 & 5 & 3 & 2 & 1 & 150 \\ \hline \hline
	Цена & 20 & 12 & 4 & 2 &  \\ \hline
\end{tabular}
\end{center}
Предполагается, что нормы затрат постоянны и цены постоянны.

Постройте модель оптимизации производства и решите её численно 
(MS Excel/Python).
\end{exercise}

\begin{exercise}
Фирма производит три товара и использует для производства два ресурса.
Норма затрат ресурсов, количество ресурсов и прибыль от каждой единицы 
товара приведены в таблице
\begin{center}
\begin{tabular}{|c|c|c|c||c|}
	\hline 
	& Товар 1 & Товар 2 & Товар 3 & Количество \\
	& & & & ресурса \\
	\hline
	Ресурс 1 & 2 & 1 & 5 & 100 \\ \hline
	Ресурс 2 & 4 & 2 & 3 & 120 \\ \hline \hline
	Цена & 3 & 8 & 2 & \\ \hline
\end{tabular}
\end{center}
Предполагается, что нормы затрат постоянны и цены постоянны.

Постройте модель оптимизации производства и решите её численно 
(MS Excel/Python).
\end{exercise}

\begin{exercise}
Фирма производит три товара и использует для производства два ресурса.
Норма затрат ресурсов, количество ресурсов и прибыль от каждой единицы
товара приведены в таблице
\begin{center}
\begin{tabular}{|c|c|c|c||c|}
	\hline
	& Товар 1 & Товар 2 & Товар 3 & Количество  \\
	& & & & ресурса \\
	\hline
	Ресурс 1 & 2 & 1 & 5 & 100 \\ \hline
	Ресурс 2 & 5 & 2 & 5 & 220 \\ \hline \hline
	Цена & 3 & 8 & 2 & \\ \hline
\end{tabular}
\end{center}
Предполагается, что нормы затрат постоянны и цены постоянны.

Постройте модель оптимизации производства и решите её численно 
(MS Excel/Python).
\end{exercise}
	

% \begin{exercise}
% Фирма производит три товара и использует для производства два ресурса.
% По плану первого товара нужно произвести не менее 100 единиц, второго --
% не менее 120, третьего -- не менее 150 ед.
% Норма затрат ресурсов и цена на ресурсы приведены в таблице
% \begin{center}
%  \begin{tabular}{|c|c|c||c|}
%   \hline 
%   & Ресурс 1 & Ресурс 2 & План \\
%   \hline
%   Товар 1 & 2 & 1 &  100 \\ \hline
%   Товар 2 & 4 & 3 &  120 \\ \hline
%   Товар 3 & 3 & 5 &  150 \\ \hline \hline
%   Цена & 3 & 6 & \\ \hline
%  \end{tabular}
% \end{center}
% Предполагается, что нормы затрат постоянны и цены экзогенны.
% \begin{enumerate}[i)]
%  \item Постройте модель оптимизации затрат ресурсов.
%  \item Постройте двойственную задачу.
%  \item Найдите оптимального количество ресурсов.
%  \item Найдите решение двойственной задачи и дайте
%  экономическую интерпретацию этого решения.
% \end{enumerate}
% \end{exercise}

\begin{exercise}%[\textbf{11 баллов}]
Фирма <<Московия>> заключила контракт с компанией АЛРОСА на покупку
промышленного золота для его реализации в пяти городах в объеме:
Самара -- 80 кг, Москва -- 260 кг, Ростов-на-Дону -- 100 кг,
Санкт-Петербург -- 140 кг, Нижний Новгород -- 120 кг. Компания
АЛРОСА располагает тремя месторождениями: <<Мирное>>, <<Удачный>> и
<<Полевое>>, которые планируют за год выработать соответственно
200, 250 и 250 кг золота.

Постройте модель оптимизации фрахта специализированного транспорта,
обеспечивающего полное удовлетворение заявок покупателя, при
заданной системе тарифов (на 1 кг)
\begin{center}\small
	\begin{tabular}{|c|c|c|c|c|c|}
	\hline
	% after \\: \hline or \cline{col1-col2} \cline{col3-col4} ...
	& Самара & Москва & Ростов-на-Дону & С.-Пб. & Н. Новгород \\ \hline
	<<Мирное>> & 7 & 9 & 15 & 4 & 18\\
	<<Удачный>> & 13 & 25 & 8 & 15 & 5 \\
	<<Полевое>> & 5 & 11 & 6 & 20 & 12\\
	\hline
	\end{tabular}
\end{center}
Найдите \textbf{численно} оптимальное решение.
\end{exercise}

\begin{exercise}
Московский филиал <<The Coca-Cola Company>>, выпускающей напитки
приблизительного равного спроса (Sprite, Coca-Cola, Fanta),
складируемых в разных местах, должен поставить свою продукцию в
четыре крупных супермаркета: <<Ашан>>, <<Карусель>>, <<Седьмой
Континент>> и <<Арбатский>>. Каждая упаковка содержит 12 банок
емкостью 0.33 литра. Тарифы на доставку, объемы запасов и заказы на
продукцию приведены в таблице.
\begin{center}\footnotesize
	\begin{tabular}{|c|c|c|c|c|c|}
	\hline
	% after \\: \hline or \cline{col1-col2} \cline{col3-col4} ...
	& \multicolumn{4}{|c|}{Супермаркеты} & \\ \hline
	Склады & Ашан & Карусель & Перекрёсток& Дикси &
	Запасы, уп.
	\\ \hline
	Coca-Cola & 6 & 4 & 9 & 5 & 400 \\
	Sprite & 5 & 7 & 8 & 6 & 300 \\
	Fanta & 9 & 4 & 6 & 7 & 200 \\ \hline
	Заказы, уп. & 150 & 250 & 150 & 350 &  \\
	\hline
	\end{tabular}
\end{center}
Постройте оптимизационную модель плана поставок напитков в
супермаркеты. Найдите \textbf{численно} оптимальное решение.
\end{exercise}

\begin{exercise}
Коммерческое предприятие реализует три группы товаров A, B и C.
Плановые нормативы затрат ресурсов (на 1 тыс рублей товарооборота),
доход от продажи товаров (на 1 тыс. рублей товарооборота)
приведены в таблице
\begin{center}\small
	\begin{tabular}{|l|c|c|c|c|}
	\hline
	& \multicolumn{3}{|c|}{Нормы затрат} & \\ \hline
	% after \\: \hline or \cline{col1-col2} \cline{col3-col4} ...
	Ресурсы & A & B & C & Объем ресурсов \\ \hline
	Рабочее время продавцов & 0.1 & 3 & 0.4 & 1100 \\
	Площадь торговых залов & 0.05 & 0.2 & 0.02 & 120 \\
	Площадь складских помещений & 3 & 0.02 & 2 & 8000 \\ \hline
	Доход & 3 & 1 & 4 &  \\
	\hline
	\end{tabular}
\end{center}
Постройте модель оптимизации для получения максимального дохода.
Найдите \textbf{численно} оптимальное решение.
\end{exercise}

% \begin{exercise}
% Для поддержания нормальной жизнедеятельности человеку ежедневно
% необходимо потреблять 118г белков, 56г жиров, 500г углеводов, 8г
% минеральных солей. Количество питательных веществ, содержащихся в
% 1кг имеющихся в магазине продуктов питания, а также их стоимость
% приведены в таблице
% \begin{center}
% {\small
% \begin{tabular}{|l|c|c|c|c|c|c|c|c|}
%   \hline
%    & \multicolumn{7}{|c|}{Содержание в 1 кг продуктов} &  \\ \hline
%   % after \\: \hline or \cline{col1-col2} \cline{col3-col4} ...
%    & мясо & рыба & молоко & масло & сыр & крупа & картофель & Нормы
%    \\ \hline
%   Белки, г & 180 & 190 & 30 & 70 & 260 & 130 & 21 & 118 \\
%   Жиры, г & 20 & 3 & 40 & 865 & 310 & 30 & 2 & 56 \\
%   Углеводы, г & 0 & 0 & 50 & 6 & 20 & 650 & 200 & 500 \\
%   Мин. соли, г & 9 & 10 & 7 & 12 & 60 & 20 & 70 & 8 \\ \hline
%   Стоимость, кг & 1.9 & 1.0 & 0.28 & 3.4 & 2.9 & 0.56 & 0.1 &  \\
%   \hline
% \end{tabular}
% }
% \end{center}
% Требуется составить модель оптимизации суточного рациона,
% содержащего не менее суточной потребности человека в белках, жирах,
% углеводах, минеральных солях и обеспечивающего минимальную стоимость
% продуктов.
% \end{exercise}

\begin{exercise}
Три нефтеперерабатывающих завода с (ежедневной) производительностью
6, 5 и 8 млн.т бензина снабжают три бензохранилища, (ежедневно)
потребность которых составляет  4, 8 и 7 млн. т бензина соответственно.
Бензин транспортируется в бензохранилища по бензопроводу. Стоимость
транспортировки составляет 0.3 руб за 1000 т на один км длины бензопровода.
В таблице приведены расстояния в км между заводами и хранилищами.
Отметим, что первый нефтеперерабатывающий завод не связан бензопроводом
с третьим бензохранилищем.
\begin{center}%\small
	\begin{tabular}{|c|c|c|c|c|} \hline
	& \multicolumn{3}{|c|}{Хранилища} & \\ \hline
 	 & 1 & 2 & 3 & Объем \\ \hline
 	 & 120 & 180 & --- & 6 \\ \hline
	2 & 300 & 100 & 80 & 5 \\ \hline
	3 & 200 & 250 & 120 & 8 \\ \hline
	Вместимость & 4& 8 & 7 & \\ %\hline
	хранилища & & & & \\ \hline
	\end{tabular}
\end{center}
Постройте оптимизационную модель транспортировки бензина.
Найдите \textbf{численно} оптимальное решение.
\end{exercise}

\section{Введение в теорию игр}

% !TEX root = exercise-book-opt.tex

\subsection{Игры с нулевой суммой}



\begin{exercise}%[\textbf{5 баллов} за каждый пункт]
Двое играют на деньги, одновременно называя одно из чисел 1 или 2, 
и потом считая сумму $S$. Если $S$ четная, то первый выигрывает у второго $S$ долларов,
если $S$ нечетная, то второй выигрывает у первого $S$ долларов. 
\begin{enumerate}
	\item Постройте платежную матрицу (матрицу полезностей) каждого из игроков. 
	Будет ли эта игра игрой с нулевой суммой? % Ответ обоснуйте.
	\item Будут ли в этой игре доминирующие стратегии? % Ответ поясните.
	\item Будут ли положения равновесия по Нэшу в чистых стратегиях? % Ответ обоснуйте.
	\item Предположим, что игроки следуют смешанным стратегиям 
	\(P^\top=\begin{pmatrix} 0.3 & 0.7\end{pmatrix}\) и \(Q^\top=\begin{pmatrix} 0.25 & 0.75\end{pmatrix}\).
	Вычислите ожидаемый выигрыш каждого из игроков.
	\item Найдите положения равновесия по Нэшу в смешанных стратегиях.
	% \item Дайте интерпретацию равновесных (по Нэшу) стратегий.
	\item Найдите ожидаемый выигрыш (полезность) каждого из игроков
	в положении равновесия по Нэшу в смешанных стратегиях.
\end{enumerate}
\end{exercise}

\begin{exercise}
Решите предыдущую задачу при условии, что игроки называю одно из 
следующих чисел: 1, 2 или 3.
\end{exercise}

\begin{exercise}
Рассмотрим антагонистическую игру с матрицей
\[
	\begin{pmatrix}
	-2 & 2 & -1 & 0 & 1 \\
	2 & 3 & 1 & 2 & 2 \\
	3 & -3 & 2 & 4 & 3 \\
	-2 & 1 & -2 & -1 & 0
	\end{pmatrix}
\]
\begin{enumerate}
	\item Найдите верхнюю и нижнюю цену игры.
	\item Существует ли равновесие в чистых стратегиях? Ответ поясните.
	\item Можно ли уменьшить размер платежной матрицы игры?
	% Как называется этот способ и в чем он состоит?
	% \item Решите данную задачу геометрическими способом задачи $2\times n$.
	\item Найдите равновесие Нэша путем сведения к задаче линейного
	программирования % (решить графически).
\end{enumerate}
\end{exercise}

\begin{exercise}
Примените к платежным матрицам операцию доминирования. Проведите анализ игры до доминирования и 
после операции доминирования. Найдите равновесие Нэша и цену игры
\begin{align*}
	a)&\;\begin{pmatrix} 2 & 1& 0 & -1 & 4 \\ 3 & 4 & 1 & 1 & 4 \\
	-1 & 0 & 2 & -4 & 1 \\ 2 & 2 & 3 & 1 & 3 \\ 4 & 5 & 3 & 1 & 2 \end{pmatrix} &
	b)&\; \begin{pmatrix} 1 & 4 & 0 & -3 & 2 \\ 3 & 3 & 2 & 4  & 1 \\ 2 & 5 & 1 & 2 & 3 \\ 
	3 & 4 & -1 & 0 & 2 \\ 2 & 2 & 1 & 1 & 0 \end{pmatrix} \\
	c)&\; \begin{pmatrix} 4 & 3 & 3 & 4 & 4 \\ 3 & -1 & -5 & 1 & 5 \\ 8 & 2 & -6 & 0 & -5 \\ 
	2 & 0 & 1 & 4 & 5 \\ 2 & 1 & 3 & 5 & 6 \\ 4 & 4 & 3 & 6 & 5 \end{pmatrix}
\end{align*}
\end{exercise}

\begin{exercise}
Рассмотрим антагонистическую игру с матрицей
\[
	\begin{pmatrix}
	2 & -1 \\ -2 & 1
	\end{pmatrix}
\]
\begin{enumerate}
	\item Рассмотрим  смешанные стратегии игроков 
	\begin{align*}
		P^\top&=\begin{pmatrix} 0.4 & 0.6 \end{pmatrix} &
		Q^\top&=\begin{pmatrix} 0.8 & 0.2 \end{pmatrix}
	\end{align*}
	Найдите ожидаемые выигрыши каждого из игроков
	\item Найдите положения равновесия по Нэшу и цену игры.
\end{enumerate}
\end{exercise}

\begin{exercise}
Найти равновесие (по Нэшу) в смешанных стратегиях и цену игры в
игре с нулевой суммой
\[
	\begin{pmatrix}
	-20 & 2 & 22 & -15 \\ 20 & -8 & -11 & 0
    \end{pmatrix}
\]
\end{exercise}

\begin{exercise}
Для антагонистической игры с матрицей
\begin{align*}
	a)&\;\begin{pmatrix} 4 & 2 & 1 & 5 \\ 2 & 3 & 6 & 3 \end{pmatrix} &
	b)&\; \begin{pmatrix} 2 & 4 \\ 0 & 5 \\ 2 & 6 \\ 3 & -4 \\ 1 & 5 \\ 3 & -1\end{pmatrix} &
	c)&\; \begin{pmatrix} -2 & 3 & 4 & 1 & 3 \\ 6 & -5 & 3 & 3 & -1 \end{pmatrix}
\end{align*}
найдите  равновесие Нэша.
\end{exercise}

\begin{exercise}
Игра <<вооружение помехи>>. Сторона $A$ располагает тремя видами вооружений
$\StrategyA_1,\StrategyA_2,\StrategyA_3$, а сторона $B$ -- тремя видами помех 
$\StrategyB_1,\StrategyB_2,\StrategyB_3$. Вероятность решения
боевой задачи стороной $A$ при различных видах вооружения и помех задана матрицей
\begin{center}
	\begin{tabular}{|c|c|c|c|}\hline
	& $\StrategyB_1$ & $\StrategyB_2$ & $\StrategyB_3$ \\ \hline
	$\StrategyA_1$ & 0.8 & 0.2 & 0.4 \\ \hline
	$\StrategyA_2$ & 0.4 & 0.5 & 0.6 \\ \hline
	$\StrategyA_3$ & 0.1 & 0.7 & 0.3 \\ \hline
	\end{tabular}
\end{center}
Сторона $A$ стремиться решить боевую задачу, сторона $B$ -- воспрепятствовать этому.
\begin{itemize}
	\item Найдите верхнюю и нижнюю цену игры. Будут ли в этой игре положения равновесия
	(по Нэшу) в чистых стратегиях?
\end{itemize}
Для удобства записи умножим матрицу на 10.
\begin{itemize}
	\item Напишите пару двойственных задач для нахождения равновесия в смешанных стратегиях.
\end{itemize}
Пусть известны оптимальные решения двойственных задач линейного программирования:
для игрока $A$
\begin{align*}
	x_1&=\frac{1}{32} & x_2&=\frac{3}{16} & x_3&=0
\end{align*}
для игрока $B$
\begin{align*}
	y_1&=\frac{3}{32} & y_2&=\frac{4}{32} & y_3&=0
\end{align*}
\begin{itemize}
	\item Найдите оптимальные стратегии каждого из игроков и цену игры.
\end{itemize}
\end{exercise}

\begin{exercise}
Полковник Блотто командует тремя отрядами. Перед ним три высоты.
Он должен решить, сколько отрядов послать на захват каждой высоты.
Его противник, граф Балони, также имеет в подчинении три отряда и должен принять такое же решение. 
Если на одной из высот у одного противника есть численное превосходство, то он захватывает эту высоту. 
Если нет, то высота остается нейтральной территорией. 
Выигрыш каждого игрока равен количеству захваченных им высот, 
минус количество высот, захваченных противником. 
Постройте матрицу игры и найдите положение равновесия по Нэшу в этой игре.
\end{exercise}


\subsection{Игры с ненулевой суммой}

\begin{exercise}
Рассмотрим платежную матрицу участников A и B некоторого парного турника,
которые придерживаются в нем одной их двух стратегий
%биматричную игру
\begin{center}
	\begin{tabular}{|c||c|c|}
	\hline
	% after \\: \hline or \cline{col1-col2} \cline{col3-col4} ...
	& $\StrategyB_{1}$ & $\StrategyB_{2}$  \\ \hline \hline
	$\StrategyA_1$ & 1, 2 & 4, 3  \\ \hline
	$\StrategyA_2$ & 3, 4 & 2, 3  \\ %\hline
	\hline
	\end{tabular}
\end{center}
\begin{enumerate}
	\item Рассмотрим  смешанные стратегии игроков 
	\begin{align*}
		P^\top&=\begin{pmatrix} 0.7 & 0.3 \end{pmatrix} &
		Q^\top&=\begin{pmatrix} 0.6 & 0.4 \end{pmatrix}
	\end{align*}
	Найдите ожидаемые выигрыши каждого из игроков
	\item Найдите положения равновесия по Нэшу в чистых и смешанных стратегиях
\end{enumerate}
\end{exercise}

\begin{exercise}
Рассмотрим биматричную игру
\begin{center}
	\begin{tabular}{|c||c|c|}
	\hline
	% after \\: \hline or \cline{col1-col2} \cline{col3-col4} ...
	& $\StrategyB_{1}$ & $\StrategyB_{2}$  \\ \hline \hline
	$\StrategyA_1$ & 4, 2 & 2, 0  \\ \hline
	$\StrategyA_2$ & 2, 2 & 3, 5  \\ %\hline
	\hline
	\end{tabular}
\end{center}
\begin{enumerate}
	\item Рассмотрим смешанные стратегии игроков 
	\begin{align*}
		P^\top&=\begin{pmatrix} 0.5 & 0.5 \end{pmatrix} &
		Q^\top&=\begin{pmatrix} 0.2 & 0.8 \end{pmatrix}
	\end{align*}
	Найдите ожидаемые выигрыши каждого из игроков
	\item Найдите положения равновесия по Нэшу в чистых и смешанных стратегиях
\end{enumerate}
\end{exercise}

\begin{exercise}
Рассмотрим биматричную игру
\begin{center}
	\begin{tabular}{|c||c|c|c|}
	\hline
	% after \\: \hline or \cline{col1-col2} \cline{col3-col4} ...
	& $\StrategyB_{1}$ & $\StrategyB_{2}$  & $\StrategyB_{3}$ \\ \hline \hline
	$\StrategyA_1$ & 4, 1 & 2, 2 & 1, 3  \\ \hline
	$\StrategyA_2$ & 2, 2 & 3, 5 & 0, 4 \\ %\hline
	\hline
	\end{tabular}
\end{center}
\begin{itemize}
	\item Рассмотрим смешанные стратегии игроков 
	\begin{align*}
		P^\top&=\begin{pmatrix} 0.6 & 0.4 \end{pmatrix} &
		Q^\top&=\begin{pmatrix} 0.2 & 0.3 & 0.5 \end{pmatrix}
	\end{align*}
	Найдите ожидаемые выигрыши каждого из игроков
	\item Найдите положения равновесия по Нэшу в чистых и смешанных стратегиях
\end{itemize}
\end{exercise}

\begin{exercise}
Рассмотрим биматричную игру
\begin{center}
	\begin{tabular}{|c||c|c|}
	\hline
	% after \\: \hline or \cline{col1-col2} \cline{col3-col4} ...
	& $\StrategyB_{1}$ & $\StrategyB_{2}$  \\ \hline \hline
	$\StrategyA_1$ & 4, 2 & 2, 0  \\ \hline
	$\StrategyA_2$ & 2, 2 & 3, 5  \\ \hline
	$\StrategyA_3$ & 3, 1 & 2, 3 \\ 
	\hline
	\end{tabular}
\end{center}
\begin{enumerate}
	\item Рассмотрим смешанные стратегии игроков 
	\begin{align*}
		P^\top&=\begin{pmatrix} 0.4 & 0.4 & 0.2 \end{pmatrix} &
		Q^\top&=\begin{pmatrix} 0.3 & 0.7 \end{pmatrix}
	\end{align*}
	Найдите ожидаемые выигрыши каждого из игроков
	\item Найдите положения равновесия по Нэшу в чистых и смешанных стратегиях
\end{enumerate}
\end{exercise}


\begin{exercise}[Дуополия Курно]
Пусть $Q_i$ -- объем выпуска, $cQ_i$ --
издержки фирмы $i=1,2$. Функция спроса имеет вид ($a>c>0$)
\[
	P(Q)=\begin{cases}
	a-Q, & Q\leq a \\
	0, & Q>a
	\end{cases}
\]
Доход фирмы определяется равенством $(P(Q_1+Q_2)-c)Q_i$.

Каждая фирма имеет две возможности: <<мелкосерийное производство>>
$Q^l=(a-c)/4$ и <<крупносерийное производство>> $Q^h=(a-c)/3$.
\begin{enumerate}
	\item Напишите биматричную игру в нормальной форме.
	\item Найдите положения равновесия (по Нэшу) в чистых и смешанных
	стратегиях.
\end{enumerate}
\end{exercise}

\begin{exercise}[Дуополия Бертрана]
Пусть на рынке минеральной воды присутствуют две конкурирующие фирмы $A$ и $B$. 
Постоянные издержки каждой из них равны 300
(вне зависимости от объема продаж). Каждая фирма
должна выбрать либо <<высокую>> цену на свою продукцию $P_h=1$, 
либо <<низкую>> цену $P_l=0.5$ (цена за бутылку). При при <<высокой>> цене на 
рынке можно продать 1000 бутылок, при <<низкой>> цене -- 2000 бутылок.
Если компании выбирают одинаковую цену, то они делят объемы продаж поровну.
Если компании выбирают разные цены, то рынок полностью захватывает компания
с более низкой ценой (другая ничего не продает).
\begin{enumerate}
	\item Постройте платежную матрицу (матрицу полезностей) каждого из игроков. 
	Будет ли эта игра игрой с нулевой суммой? % Ответ обоснуйте.
	\item Будут ли в этой игре доминирующие стратегии? % Ответ поясните.
	\item Будут ли положения равновесия по Нэшу в чистых стратегиях? % Ответ обоснуйте.
	\item Найдите положения равновесия по Нэшу в смешанных стратегиях.
	\item Дайте интерпретацию равновесных (по Нэшу) стратегий.
	\item Найдите ожидаемый выигрыш (полезность) каждого из игроков
	в положении равновесия по Нэшу в смешанных стратегиях.
\end{enumerate}
\end{exercise}

\begin{exercise}[Дуополия Бертрана]
Пусть на рынке минеральной воды присутствуют две конкурирующие фирмы $A$ и $B$. 
Постоянные издержки каждой из них равны \$5000
(вне зависимости от объема продаж). Каждая фирма
должна выбрать либо <<высокую>> цену на свою продукцию $P_h=\$2$, либо <<низкую>> цену $P_l=\$1$
(цена за бутылку). Тогда:
\begin{enumerate}
	\item при <<высокой>> цене на рынке можно продать 5000 бутылок,
	\item при <<низкой>> цене на рынке можно продать 10000 бутылок,
	\item если компании выбирают одинаковую цену, то они делят объемы продаж поровну,
	\item если компании выбирают разные цены, то рынок полностью захватывает компания
	с более низкой ценой (другая ничего не продает).
\end{enumerate}
Постройте матрицу игры. Будет ли это игра с нулевой суммой? Найдите положения равновесия
по Нэшу и цену игры.
\end{exercise}

\begin{exercise}
Рассмотрим биматричную игру
\begin{center}
	\begin{tabular}{|c||c|c|c|c|}
	 \hline
	% after \\: \hline or \cline{col1-col2} \cline{col3-col4} ...
	& $\StrategyB_{1}$ & $\StrategyB_{2}$ & $\StrategyB_{3}$ & $\StrategyB_{4}$\\ \hline \hline
	$\StrategyA_1$ & 2, 2 & 1, 1 & 1, 3 & 2, 1\\ \hline
	$\StrategyA_2$ & 2, 4 & 1, 3 & 1, 2 & 4, 2\\ \hline
	$\StrategyA_3$ & 3, 2 & 2, 3 & 2, 5 & 3, 1 \\ \hline
	$\StrategyA_4$ & 4, 1 & 1, 4 & 3, 1 & 2, 2 \\
	\hline
	\end{tabular}
\end{center}
Проведите процедуру последовательного удаления доминируемых стратегий.
Будут ли в этой игре положения равновесия по Нэшу и по Парето в чистых стратегиях?
\end{exercise}

\begin{exercise}
У двух авиапассажиров, следовавших одним рейсом, пропали чемоданы. 
Авиакомпания готова возместить ущерб каждому пассажиру. 
Для того, чтобы определить размер компенсации, каждого пассажира просят сообщить, 
во сколько он оценивает содержимое своего чемодана. 
Каждый пассажир может назвать сумму не менее \$90 и не более \$100. 
Условия компенсации таковы: если оба сообщают одну и ту же сумму, 
то каждый получит эту сумму в качестве компенсации. 
Если же заявленный одним из пассажиров ущерб окажется меньше, 
чем заявленный ущерб другого пассажира, то каждый пассажир получит компенсацию, 
равную меньшей из заявленных сумм. При этом тот, кто заявил меньшую сумму,
получит дополнительно \$2, тот, кто заявил большую сумму — дополнительно потеряет два доллара.
Постройте матрицу игры и найдите положение равновесия по Нэшу в этой игре.
\end{exercise}

\begin{exercise}
Винни Пух и Пятачок решают, к кому они пойдут в гости --- к Пуху
(стратегия 1), Пятачку (стратегия 2) или к Кролику (стратегия 3).
Друзья могут пойти как вместе так и по раздельности. С точки
зрения Пуха, удовольствие от похода в гости оценивается так: 1, 2
и 4 соответственно. С точки зрения Пятачка удовольствия таковы: 3,
2 и 2 соответственно. Присутствие спутника (совместный поход в
гости) добавляет одну единицу удовольствия каждому. Найти
равновесие в этой игре.
\end{exercise}

\begin{exercise}
Две фирмы делят рынок. У каждой фирмы три стратегии --- выставить
низкую цену, среднюю цену и высокую цену $p_1=2$, $p_2=4$,
$p_3=7$. Предположим, что спрос на рынке составляет $100$ единиц
продукции, при равенстве цен он делится поровну между фирмами,
если одна фирма выставила высокую цену, а другая среднюю, то спрос
составляет $30$ и $70$ единиц соответственно. При средней и низкой
цене спрос также делится на $30$ и $70$, а при высокой и низкой
он составляет $10$ и $90$ единиц соответственно. Найти положение
равновесия по Нэшу.
\end{exercise}

\begin{exercise}
Две радиостанции делят рынок, выбирая из двух форматов вещания: новости и музыка.
Целевая аудитория каждого из форматов равна 38\% и 62\% соответственно. Если радиостанции
выбирают одинаковый формат вещания, то они делят целевую аудитория пополам.
Если Если радиостанции выбирают разный формат вещания, то каждая ``захватывает''
всю целевую аудиторию своего формата вещания. Выигрыш радиостанции -- её доля аудитории.
\begin{enumerate}
	\item Запишите игру в нормальной форме. 
	Будет ли эта игра игрой с нулевой суммой? % Ответ обоснуйте.
	\item Найдите положения равновесия по Нэшу в чистых и смешанных стратегиях
	и выигрыши игроков в равновесии.
\end{enumerate}
\end{exercise}

% \subsection{Байесовские игры}

% \begin{exercise}
% Двое лыжников едут, разогнавшись, по одной лыжне навстречу друг другу. 
% Каждый должен решить, уступать лыжню или нет. Никто не хочет уступать друг другу дорогу. 
% Однако если никто лыжню не уступит, то произойдет столкновение. 

% Первый лыжник считает, что второй лыжник с вероятностью 0.5 — принципиальный, 
% то есть получает большой моральный штраф, 
% если будет уступать лыжню. Тогда игра имеет вид:
% \begin{center}
% 	\begin{tabular}{c|c|c|}
% 	& Уступить 2 & Не уступить 2 \\ \hline
% 	Уступить 1 & $0; -4$ & $-1;1$ \\ \hline
% 	Не уступить 1& $1; -5$ & $-4; -4$ \\ \hline
% 	\end{tabular}
% \end{center}
% Также первый лыжник считает, что второй лыжник с вероятностью 0.5 — обычный добродушный лыжник. 
% Тогда игра имеет вид:
% \begin{center}
% 	\begin{tabular}{c|c|c|}
% 	& Уступить 2 & Не уступить 2 \\ \hline
% 	Уступить 1 & $0; 0$ & $-1;1$ \\ \hline
% 	Не уступить 1& $1;-1$ & $-4; -4$ \\ \hline
% 	\end{tabular}
% \end{center}
% Первый лыжник — добродушный. Найдите равновесия Байеса-Нэша
% \end{exercise}

% \begin{exercise}
% В деревне живет 11 лыжников, некоторые из которых имеют добродушный нрав, а некоторые — злобный. 
% Всем жителям известно, что добродушных лыжников в деревне 8 человек, а остальные лыжники злобные. 

% 1 января ближе к вечеру один из лыжников решил прокатиться по единственной в округе лыжне. 
% При подъезде к очередному глубокому оврагу он заметил на противоположной стороне другого лыжника, 
% однако из-за расстояния не смог его узнать. Каждый спортсмен независимо от другого решет, 
% съезжать ли ему в овраг или пропустить другого лыжника. 
% Если оба решат съехать, то произойдет столкновение и каждый получит платеж -4. 

% Добродушный лыжник получает моральное удовлетворение от пропуска другого лыжника в размере 1, 
% если другой воспользовался преимуществом, и ничего не получает, если другой преимуществом не воспользовался. 
% Если же пропустили его и он этим воспользовался, то он испытывает угрызения совести в размере -1. 

% Злобный лыжник испытывает неудовольствие от пропуска другого лыжника в размере -1, 
% если другой воспользовался преимуществом, и ничего не получает, 
% если другой преимуществом не воспользовался. Если же пропустили его и он этим воспользовался, 
% то он испытывает удовольствие в размере 1.

% Найдите равновесия Байеса-Нэша
% \end{exercise}

% \begin{exercise}
% Коля и Маша играют в упрощенную версию покера. У каждого игрока может быть на руках либо 
% хорошая комбинация карт, либо плохая. Оба игрока должны одновременно решить, 
% повышать ставку или нет. Коля по неосторожности <<засветил>> свои карты и теперь Маша знает, 
% что у Коли хорошая карта. Коля достоверно не знает, какие карты у Маши, но по его оценкам, 
% с вероятностью 0.7 у нее плохая карта, а с вероятностью 0.3 -- хорошая. 
% Каждый из игроков знает свою карту. Если у Маши плохая карта, то игра для Коли выглядит следующим образом:
% \begin{center}
% 	\begin{tabular}{c|c|c|}
% 	& Повышать 2 & Не повышать 2 \\ \hline
% 	Повышать 1 & $2; -2$ & $1;-1$ \\ \hline
% 	Не повышать 1& $-1; 1$ & $1; -1$ \\ \hline
% 	\end{tabular}
% \end{center}
% Если у Маши хорошая карта, то игра имеет вид:
% \begin{center}
% 	\begin{tabular}{c|c|c|}
% 	& Повышать 2 & Не повышать 2 \\ \hline
% 	Повышать 1 & $0; 0$ & $1;-1$ \\ \hline
% 	Не повышать 1& $-1; 1$ & $0; 0$ \\ \hline
% 	\end{tabular}
% \end{center}
% Найдите все равновесия Байеса-Нэша в этой игре.
% \end{exercise}

% \begin{exercise}
% После очередного сражения командующий войсками страны А предлагает командующему войсками 
% страны В капитулировать. О численности войск и состоянии армии противника после боя командирам неизвестно.
% Командующий войсками страны В, по численности армии, изначально превосходящей страну А, 
% предполагает, что с вероятностью $0.7$ командующий войсками страны А может блефовать и 
% сам находится в невыгодном положении. Следующий бой скоро и командирам нужно решить, что делать -- 
% продолжить сражение или капитулировать. При этом, до командующего войсками страны А доходили слухи, 
% что командующий войсками страны В -- темный маг, который может призвать нечисть и выиграть битву. 
% Так как это, по его мнению, невозможно, то он относится к такой информации скептически и верит ей лишь на 15\%. 
% Независимо от того, в каком положении находится командир А, если он капитулирует, то получает $-100$ (минус 100), 
% если противник, кем бы он ни был, собирался атаковать, и $-300$ (минус 300), если собирался сдаться. 
% Если командир А, находясь в проигрышном положении, идет в бой с готовым атаковать противником, 
% то теряет армию и сам умирает, следовательно получает $-500$ (минус 500), если бой был против темного мага и $-800$
% (минус 800),  если против обычного человека. Все немного лучше, если он находится в выигрышном положении -- тогда, 
% если в бою его атакует обычный противник, то оба в таком исходе получают 0 т.к. сражение произошло в ничью; 
% если же войска противника вел темный маг, то командир А получает также $-500$ (минус 500). 
% Если командир А в выигрышном положении нападает получает 500, если противник сдался, будучи темным магом, и 300, 
% если противник темным магом не был. Если его положение проигрышное, то к этим платежам прибавляется +200. 
% Для капитана В, если он темный маг, при вступлении в бой все нипочем, он получает 700, но если по каким-то причинам 
% он капитулирует, то его платеж равен $-1000$ (минус 1000). Если В -- обычный человек, и идет в бой против противника в
% проигрышном положении, то получает 100, если тот дает бой, и 500, если капитулирует; если противник в выигрышном
% положении, то 0 (как уже было сказано) и 300 соответственно. Если обычный капитан В решает капитулировать, то
% получает $-100$ (минус 100), если противник в проигрышном положении идет в бой, и $-300$ (минус 300), 
% если капитулирует. К этим платежам прибавляется по 100, если противник в выигрышном положении.
% \begin{enumerate}
% 	\item Формализуйте игру
% 	\item Напишите, как игра выглядит с точки зрения каждого из участников.
% 	\item Найдите все равновесия Байеса-Нэша в данной игре.
% \end{enumerate}
% \end{exercise}

\appendix

\section{Приложение}

% !TEX root = exercise-book-opt.tex

\subsection{Симметричные матрицы}

Пусть \(\matrixA\) -- (\(n\times n\)) симметричная матрица.

\begin{teorema}[Критерий Сильвестра]\label{SylvesterCiterion}
Пусть $\matrixA$ -- симметричная матрица и
$\Delta_1,\ldots,\Delta_n$ последовательность ее угловых миноров:
\begin{align*}
	\Delta_1&=a_{11} & \Delta_2&=\det\begin{pmatrix} a_{11} & a_{12} \\ a_{21} & a_{22} \end{pmatrix} & 
	&\ldots & \Delta_n&=\det\matrixA
\end{align*}
Тогда
\begin{enumerate}
	\item $\matrixA>0\iff\Delta_i>0$, $i=1,\ldots,n$.
	\item $\matrixA<0\iff(-1)^i\Delta_i>0$, $i=1,\ldots,n$.
	\item если знаки миноров не удовлетворяют предыдущим пунктам, 
	то матрица не знакоопределена
\end{enumerate}  
\end{teorema}

\begin{propuesta}\label{2times2definitness}
Пусть $\matrixA$ -- симметричная $2\times 2$ 
Тогда
\begin{align*}
	\matrixA\geq0 &\iff \begin{matrix} a_{11} \\ a_{22} \end{matrix} \geq0,\;\det\matrixA\geq0\\
	\matrixA\leq0  &\iff \begin{matrix} a_{11} \\ a_{22} \end{matrix} \leq0,\;\det\matrixA\geq0.
\end{align*}    
\end{propuesta}

Для  $3\times3$ матрицы обозначим центральные миноры 
\begin{align*}
	\Minor_{(12)}&=\det\begin{pmatrix} a_{11} & a_{12} \\ a_{21} & a_{22} \end{pmatrix} \\
	\Minor_{(13)}&=\det\begin{pmatrix} a_{11} & a_{13} \\ a_{31} & a_{33} \end{pmatrix} \\
	\Minor_{(23)}&=\det\begin{pmatrix} a_{22} & a_{23} \\ a_{32} & a_{33} \end{pmatrix} 
\end{align*}

\begin{propuesta}\label{3times3definitness}
Пусть $\matrixA$ -- симметричная $3\times 3$. Тогда
\begin{align*}
	\matrixA\geq 0 &\iff \begin{matrix} a_{11} \\ a_{22} \\ a_{33} \end{matrix}\geq0,\;
	\begin{matrix} \Minor_{(12)} \\ \Minor_{(13)} \\ \Minor_{(23)} \end{matrix}\geq0,\;
	\det\matrixA\geq0\\
	\matrixA\leq 0 &\iff \begin{matrix} a_{11} \\ a_{22} \\ a_{33} \end{matrix}\leq0,\;
	\begin{matrix} \Minor_{(12)} \\ \Minor_{(13)} \\ \Minor_{(23)} \end{matrix}\geq0,\;
	\det\matrixA\leq0.
\end{align*}
\end{propuesta}

\subsection{Выпуклые функции}

Пусть числовая функция \(f\) определена на % выпуклом множестве 
\(\Domain(f)\subset \R^n\)

\begin{teorema}
Дважды непрерывно дифференцируемая функция $f$ выпукла $\iff$ 
$\Hessian_f(\vectx)\geq0$ %(как симметричная матрица) 
для всех $\vectx\in\Domain(f)$.

Если $\Hessian_f(\vectx)>0$ для всех $\vectx\in\Domain(f)$, 
то функция строго выпукла на $\Domain(f)$.
\end{teorema}

\begin{col}
Дважды непрерывно дифференцируемая функция $f$ вогнута $\iff$ 
$\Hessian_f(\vectx)\leq0$ % (как симметричная матрица) 
для всех $\vectx\in\Domain(f)$.

Если $\Hessian_f(\vectx)<0$ для всех $\vectx\in\Domain(f)$, 
то функция строго вогнута
на $\Domain(f)$.
\end{col}

\begin{remark}
Знак гессиана проверяем по критерию Сильвестра 
\ref{SylvesterCiterion} или
используем Предложения \ref{2times2definitness}, \ref{3times3definitness}
\end{remark}


\subsection{Функция Лагранжа для ограничений равенства}

Рассмотрим задачи оптимизации с ограничениями равенства

\begin{align*}
	& \begin{gathered}
		\max f(\vectx) \\
		s.t.\left\{\begin{aligned}
			g_1(\vectx)&=c_1 \\ &\vdots \\ g_k(\vectx)&=c_k
		\end{aligned}\right.
	\end{gathered} &
	& \begin{gathered}
		\min f(\vectx) \\
		s.t.\left\{\begin{aligned}
			g_1(\vectx)&=c_1 \\ &\vdots \\ g_k(\vectx)&=c_k
		\end{aligned}\right.
	\end{gathered}
\end{align*}
Функция Лагранжа для этих задач
\[
	\Lagrange(\vectx,\vectlambda)=f(\vectx)-\sum_{j=1}^k \lambda_jg_j(\vectx)
\]
\textbf{Необходимые условия} (локального) условного экстремума
\[
	\left\{\begin{aligned}
		\Lagrange'_{x_i}&=0 & i&=1,\ldots,n\\
		g_j(\vectx) &= c_j & j&=1,\ldots,k
	\end{aligned}\right.
\]
Гессиан для функции Лагранжа (симметричная матрица)
\[
	\underset{(n+k)\times(n+k)}{\BordHessian_\Lagrange}=\begin{pmatrix}
		\frac{\partial^2 \Lagrange}{\partial x_i\partial x_j} & | &\frac{\partial^2 \Lagrange}{\partial x_i\partial \lambda_l} \\
		-- & + & -- \\
		\frac{\partial^2 \Lagrange}{\partial \lambda_s\partial x_j} & | &
		\frac{\partial^2 \Lagrange}{\partial \lambda_s\partial \lambda_l}
   \end{pmatrix}
\]
Из определения функции Лагранжа
\begin{itemize}
	\item \(\frac{\partial^2 \Lagrange}{\partial \lambda_s\partial \lambda_l}=0\)
	\item \(\frac{\partial^2 \Lagrange}{\partial \lambda_s\partial x_j}=-\frac{\partial g_s}{\partial x_j}\)
\end{itemize}
Явный вид гессиана
\begin{equation}\label{BorderedHessian}
	\underset{(n+k)\times(n+k)}{\BordHessian_\Lagrange}=
%    \begin{pmatrix}
%      \frac{\partial^2 \Lagrange}{\partial x_i\partial x_j} & | &\frac{\partial^2 \Lagrange}{\partial x_i\partial \lambda_l} \\
%      -- & + & -- \\
%      \frac{\partial^2 \Lagrange}{\partial \lambda_s\partial x_j} & | &
%      \frac{\partial^2 \Lagrange}{\partial \lambda_s\partial \lambda_l}
%    \end{pmatrix}=\\
	\begin{pmatrix} 
		\frac{\partial^2 \Lagrange}{\partial x_1\partial x_1} &  \cdots &
		\frac{\partial^2 \Lagrange}{\partial x_1\partial x_n} & -\frac{\partial g_1}{\partial x_1} & \cdots & 
		-\frac{\partial g_k}{\partial x_1} \\ 
		\frac{\partial^2 \Lagrange}{\partial x_2\partial x_1} &  \cdots &
		\frac{\partial^2 \Lagrange}{\partial x_2\partial x_n} & -\frac{\partial g_1}{\partial x_2} & \cdots & 
		-\frac{\partial g_k}{\partial x_2}\\
		\vdots & \ddots & \vdots & \vdots & \ddots & \vdots \\
		\frac{\partial^2 \Lagrange}{\partial x_n\partial x_1} & \cdots &
		\frac{\partial^2 \Lagrange}{\partial x_n\partial x_n}& -\frac{\partial g_1}{\partial x_n} & \cdots & 
		-\frac{\partial g_k}{\partial x_n}\\
		-\frac{\partial g_1}{\partial x_1} & \cdots &
		-\frac{\partial g_1}{\partial x_n} & 0 & \cdots & 0\\
		\vdots & \ddots & \vdots & \vdots & \ddots & \vdots\\
		-\frac{\partial g_k}{\partial x_1} &  \cdots &
		-\frac{\partial g_k}{\partial x_n} & 0 & \cdots & 0\\
	\end{pmatrix}
\end{equation}
Пусть $\Minor_{i}$ ($i=1,...,n+k$) -- главный минор матрицы $\BordHessian_\Lagrange$, 
образованный строками и столбцами с индексами $i,i+1,...,n+k$.

\begin{teorema}[Достаточные условия минимума]\label{EqualityConstraintMinSufficientCondition}
Пусть в точке $\hat{\vectx}$  ранг матрицы $(\frac{\partial g_j}{\partial x_i})$ максимален и 
эта точка удовлетворяет необходимым условия экстремума.\\
Тогда достаточным условием локального минимума является выполнение неравенств
\begin{align}\label{EqualityConstraintMinSignSufficientCondition}
	(-1)^k\Minor_i(\hat{\vectx},\hat{\vectlambda})&>0 & 
	i&=1,\ldots,n-k.
\end{align}
\end{teorema}
\begin{remark}
Условие \eqref{EqualityConstraintMinSignSufficientCondition} означает, что
все миноры $\Minor_1,\ldots,\Minor_{n-k}$ имеют знак $(-1)^k$.
\end{remark}

\begin{teorema}[Достаточные условия максимума]\label{EqualityConstraintMaxSufficientCondition}
Пусть в точке $\hat{\vectx}$  ранг матрицы $(\frac{\partial g_j}{\partial x_i})$ максимален и 
эта точка удовлетворяет необходимым условия экстремума.\\
Тогда достаточным условием наличия локального максимума является выполнение неравенств
\begin{align}\label{EqualityConstraintMaxSignSufficientCondition}
	(-1)^n(-1)^{i-1}\Minor_i(\hat{\vectx},\hat{\vectlambda})&>0 & 
	i&=1,\ldots,n-k.
\end{align}
\end{teorema}
\begin{remark}
Условие \eqref{EqualityConstraintMaxSignSufficientCondition} означает чередование знаков
в последовательности миноров $\Minor_1,\ldots,\Minor_{n-k}$, начиная со знака $(-1)^n$.
\end{remark}
Рассмотрим два частных случая:
\begin{itemize}
	\item \(n=2,k=1\) (две переменных при одном ограничении). Тогда 
	\(n-k=1\) и 
	\begin{align*}
		(loc)\min:&\; \Minor_1<0 & (loc)\max:&\; \Minor_1>0
	\end{align*}
	\item \(n=3,k=1\) (три переменных при одном ограничении). Тогда 
	\(n-k=2\) и 
	\begin{align*}
		(loc)\min:&\; \Minor_1,\Minor_2<0 & 
		(loc)\max:&\; \Minor_1<0,\Minor_2>0
	\end{align*}
\end{itemize}
	
\subsection{Функция Лагранжа для ограничений неравенства}

\paragraph{Задача 1}

Будем рассматривать задачу максимизации в виде
\begin{equation}\label{MaxInequalityConstraintOptmization}
	\begin{aligned} & \textcolor{red}{\max}  f(\vectx) \\ 
		s.t. & 
		\left\{\begin{aligned} g_1(\vectx) & \textcolor{red}{\leq} C_1 \\ 
		& \vdots \\ 
		g_k(\vectx) & \textcolor{red}{\leq} C_k \end{aligned}\right.
	\end{aligned} 
\end{equation}
и задачу минимизации в виде
\begin{equation}\label{MinInequalityConstraintOptmization}
	\begin{aligned} & \textcolor{red}{\min}  f(\vectx) \\ 
		 s.t. & 
		 \left\{\begin{aligned} g_1(\vectx) & \textcolor{red}{\geq} C_1 \\ 
		 & \vdots \\ 
		 g_k(\vectx) & \textcolor{red}{\geq} C_k \end{aligned}\right.
	 \end{aligned} 
\end{equation}

\begin{importante}
Согласованность знаков неравенства в ограничениях с максимизацией/минимизацией
целевой функции.
\end{importante}

\begin{definicion}
Задача оптимизации \eqref{MaxInequalityConstraintOptmization} называется 
\textbf{задачей выпуклого программирования} или \textbf{выпуклой оптимизации}, если
\begin{itemize}
	\item $f$ вогнута
	\item $g_1,\ldots,g_k$ выпуклы
\end{itemize}
\end{definicion}
\begin{definicion}
Задача оптимизации \eqref{MinInequalityConstraintOptmization} называется 
\textbf{задачей выпуклого программирования} или \textbf{выпуклой оптимизации}, если
\begin{itemize}
	\item $f$ выпукла
	\item $g_1,\ldots,g_k$ вогнуты
\end{itemize}
\end{definicion}
Обозначим
\begin{align*}
	\vectg(\vectx)&=\begin{pmatrix}
		g_1(\vectx) \\ \vdots \\ g_k(\vectx)
	\end{pmatrix} & \vectc&=\begin{pmatrix}
		C_1 \\ \vdots \\ C_k
	\end{pmatrix}
\end{align*}
Перепишем задачи \eqref{MaxInequalityConstraintOptmization},
\eqref{MinInequalityConstraintOptmization} в виде
\begin{align*}
	&\max f(\vectx) & &\min f(\vectx) \\
	s.t.\;&\vectg(\vectx)\leq \vectc  &
	s.t.\;&\vectg(\vectx)\geq \vectc 
\end{align*}
Функцией Лагранжа для задачи \eqref{MaxInequalityConstraintOptmization} и задачи
\eqref{MinInequalityConstraintOptmization} имеет вид
\[
	\Lagrange(\vectx,\vectlambda)=f(\vectx)-\sum_{j=1}^k\lambda_j(g_j(\vectx)-C_j)=
	f(\vectx)-\vectlambda^\top(\vectg(\vectx)-\vectc)
\]
Дополнительные переменные $\vectlambda=(\lambda_1,\ldots,\lambda_k)^\top$ называются
\textbf{множителями Лагранжа} (Lagrange multipliers). 
\begin{definicion}
Будем говорить, что для ограничений задачи \eqref{MaxInequalityConstraintOptmization} выполнено
\textbf{условие Слейтера}, если существует $\vectx_0\in\R^n$ такой, что
\[
	\vectg(\vectx_0)<\vectc
\]
Будем говорить, что для ограничений задачи \eqref{MinInequalityConstraintOptmization} выполнено
\textbf{условие Слейтера}, если существует $\vectx_0\in\R^n$ такой, что 
\[
	\vectg(\vectx_0)>\vectc
\]
\end{definicion}
\begin{teorema}
Пусть задача \eqref{MaxInequalityConstraintOptmization} 
является задачей выпуклого программирования и для ограничений выполнено условие Слейтера.

Тогда $\hat{\vectx}$ является решением экстремальной 
задачи тогда и только тогда, когда он удовлетворяет системе 
\begin{equation}\label{MaxInequalityConstraitNecessarySystem}
	\left\{\begin{aligned}
		\Lagrange'_{x_i}&=0 & i&=1,\ldots,n \\
		\lambda_j\Lagrange'_{\lambda_j} &=0 & j&=1,\ldots,k \\
		\textcolor{red}{\lambda_j\geq 0},\Lagrange'_{\lambda_j}&\geq0 & j&=1,\ldots,k 
	\end{aligned}\right.
\end{equation}
Кроме того, $\hat{\vectx}$ -- глобальный максимум в задаче  \eqref{MaxInequalityConstraintOptmization}.
\end{teorema}
\begin{col}%[Необходимые условия минимума]\label{MinInequalityConstraintNecessaryCondition} 
Чтобы точка $\vectx$ была локальным минимумом в задаче \eqref{MinInequalityConstraintOptmization}
необходимо, чтобы для некоторых чисел $\hat{\lambda}_1,...,\hat{\lambda}_k$ выполнялась  
система
\begin{equation}\label{MinInequalityConstraitNecessarySystem}
	\left\{\begin{aligned}
		\Lagrange'_{x_i}&=0 & i&=1,\ldots,n \\
		\lambda_j\Lagrange'_{\lambda_j} &=0 & j&=1,\ldots,k \\
		\textcolor{red}{\lambda_j\geq 0},\Lagrange'_{\lambda_j}&\leq0 & j&=1,\ldots,k
	\end{aligned}\right.
\end{equation}
Кроме того, $\hat{\vectx}$ -- глобальный минимум в задаче  \eqref{MinInequalityConstraintOptmization}.
\end{col}
Пусть $\hat{\vectx}$ -- решение
задачи \eqref{MaxInequalityConstraintOptmization} или задачи 
\eqref{MinInequalityConstraintOptmization}.
Рассмотрим это решение как функцию от $C_1,\ldots,C_k$:
\begin{align*}
	\hat{\vectx}&=\hat{\vectx}(C_1,\ldots,C_k) &
	\hat{f}&=f(\hat{\vectx})=\hat{f}(C_1,\ldots,C_k)
\end{align*}
\begin{teorema}
Оптимальное решение есть гладкая функция от $C_1,\ldots,C_k$ и выполнены равенства
\begin{align*}
	\frac{\partial \hat{f}}{\partial C_j}&=\hat{\lambda}_j & j&=1,\ldots,k
\end{align*}
\end{teorema}

\paragraph{Задача 2}

Рассмотрим теперь задачу максимизации в следующей форме
\begin{equation}\label{MaxKuhnTuckerInequalityConstraintOptmization}%\tag{max}
	\begin{aligned}
	&\textcolor{red}{\max}\, f(\vectx)  \\
	s.t. & \left\{ 
	\begin{aligned} 
	& g_j(\vectx)\textcolor{red}{\leq} C_j & j&=1,\ldots,k\\
	& \textcolor{red}{x_i\geq 0} &  i&=1,\ldots,n
	\end{aligned}\right.
	\end{aligned} 
\end{equation}
и задачу минимизации в форме
\begin{equation}\label{MinKuhnTuckerInequalityConstraintOptmization}%\tag{min}
	\begin{aligned}
		&\textcolor{red}{\min}\, f(\vectx)  \\
		s.t. & \left\{ 
		\begin{aligned} 
		& g_j(\vectx)\textcolor{red}{\geq} C_j & j&=1,\ldots,k\\
		& \textcolor{red}{x_i\geq 0} &  i&=1,\ldots,n
	\end{aligned}\right.
	\end{aligned} 
\end{equation}
(отдельно выделяем ограничения неотрицательности переменных)
\begin{definicion}
Задача оптимизации \eqref{MaxKuhnTuckerInequalityConstraintOptmization} называется 
\textbf{задачей выпуклого программирования} или \textbf{выпуклой оптимизации}, если
\begin{itemize}
	\item $f$ вогнута
	\item $g_1,\ldots,g_k$ выпуклы
\end{itemize}
Задача оптимизации \eqref{MinKuhnTuckerInequalityConstraintOptmization} называется 
\textbf{задачей выпуклого программирования} или \textbf{выпуклой оптимизации}, если
\begin{itemize}
	\item $f$ выпукла
	\item $g_1,\ldots,g_k$ вогнуты
\end{itemize}
\end{definicion}
Обозначим
\begin{align*}
	\vectg(\vectx)&=\begin{pmatrix}
		g_1(\vectx) \\ \vdots \\ g_k(\vectx)
	\end{pmatrix} & \vectc&=\begin{pmatrix}
		C_1 \\ \vdots \\ C_k
	\end{pmatrix}
\end{align*}
Перепишем задачи \eqref{MaxKuhnTuckerInequalityConstraintOptmization},
\eqref{MinKuhnTuckerInequalityConstraintOptmization} в виде
\begin{align*}
	&\max f(\vectx) & &\min f(\vectx) \\
	s.t.&\left\{\begin{aligned} \vectg(\vectx)&\leq \vectc \\ \vectx &\geq 0\end{aligned}\right. &
	s.t.&\left\{\begin{aligned} \vectg(\vectx)&\geq \vectc \\ \vectx &\geq 0\end{aligned}\right.
\end{align*}
Определим для задач \eqref{MaxKuhnTuckerInequalityConstraintOptmization},
\eqref{MinKuhnTuckerInequalityConstraintOptmization}
\textbf{функцию Лагранжа в форме Куна -- Таккера}, включив в неё только
нетривиальные ограничения
\[
	\LagrangeZ(\vectx,\vectmu)=f(\vectx)-\sum_{j=1}^k\mu_j(g_j(\vectx)-C_j)=
	f(\vectx)-\vectmu^\top (\vectg(\vectx)-\vectc)
\]
с множителями Лагранжа 
\(\vectmu=\begin{pmatrix} \mu_1 & \cdots & \mu_k \end{pmatrix}^\top \)
\begin{definicion}
Будем говорить, что для ограничений задачи \eqref{MaxKuhnTuckerInequalityConstraintOptmization} выполнено
\textbf{условие Слейтера}, если существует $\vectx_0\in\R^n$ такой, что
\begin{align*}
	\vectg(\vectx_0)&<\vectc & \vectx_0&>0
\end{align*}
Будем говорить, что для ограничений задачи \eqref{MinKuhnTuckerInequalityConstraintOptmization} выполнено
\textbf{условие Слейтера}, если существует $\vectx_0\in\R^n$ такой, что 
\begin{align*}
	\vectg(\vectx_0)&>\vectc & \vectx_0&>0
\end{align*}
\end{definicion}
\begin{remark}
Условие Слейтера относится только к системе ограничений и не касается целевой функции
\end{remark}
	
\begin{teorema}%[Kuhn -- Tucker, 1951]
Пусть задача \eqref{MaxKuhnTuckerInequalityConstraintOptmization} 
является задачей выпуклого программирования и для ограничений выполнено условие Слейтера.

Тогда $\hat{\vectx}$ является решением экстремальной 
задачи тогда и только тогда, когда он удовлетворяет системе 
\begin{equation}\label{MaxKuhnTuckerSystem}
	\left\{\begin{aligned}
	x_i \LagrangeZ'_{x_i}&=0 &  i&=1,\ldots,n \\
	\mu_j\LagrangeZ'_{\mu_j}&=0 &  j&=1,\ldots,k\\
	x_i\geq0,\; \LagrangeZ'_{x_i}&\leq0 &  i&=1,\ldots,n \\
	\mu_j\geq0,\; \LagrangeZ'_{\mu_j}&\geq 0 &  j&=1,\ldots,k
	\end{aligned}\right.
\end{equation}
Кроме того, $\hat{\vectx}$ -- глобальный максимум в задаче  \eqref{MaxKuhnTuckerInequalityConstraintOptmization}.
\end{teorema}

\begin{teorema}%[Kuhn -- Tucker, 1951]
Пусть задача \eqref{MinKuhnTuckerInequalityConstraintOptmization} 
является задачей выпуклого программирования и для ограничений выполнено условие Слейтера.

Тогда $\hat{\vectx}$ является решением экстремальной 
задачи тогда и только тогда, когда он удовлетворяет системе 
\begin{equation}\label{MinKuhnTuckerSystem}
	\left\{\begin{aligned}
	x_i\LagrangeZ'_{x_i}&=0 & i&=1,\ldots,n \\
	\mu_j\LagrangeZ'_{\mu_j}&=0 & j&=1,\ldots,k \\
	x_i\geq0,\; \LagrangeZ'_{x_i}&\geq0 & i&=1,\ldots,n \\
	\mu_j\geq0,\; \LagrangeZ'_{\mu_j}& \leq0 & j&=1,\ldots,k 
	\end{aligned}\right.
\end{equation}
Кроме того, $\hat{\vectx}$ -- глобальный минимум в задаче  \eqref{MinKuhnTuckerInequalityConstraintOptmization}.
\end{teorema}
Рассмотрим оптимальное решение задачи \eqref{MaxKuhnTuckerInequalityConstraintOptmization} и задачи 
\eqref{MinKuhnTuckerInequalityConstraintOptmization} и оптимальное значение целевой функции
как функцию от  $C_1,\ldots,C_k$: 
\begin{align*}
	\hat{\vectx}&=\hat{\vectx}(C_1,\ldots,C_k) & 
	\hat{f}&=f(\hat{\vectx})=\hat{f}(C_1,\ldots,C_k).
\end{align*}
\begin{teorema}
$\hat{\vectx}=\hat{\vectx}(C_1,\ldots,C_k)$ есть гладкая функция и выполнены равенства
\begin{align*}
	\mu_j&=\frac{\partial \hat{f}}{\partial C_j} & j&=1,\ldots,k
\end{align*}
\end{teorema}

\end{document}