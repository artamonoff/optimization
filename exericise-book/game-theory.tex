% !TEX root = exercise-book-opt.tex

\subsection{Игры с нулевой суммой}



\begin{exercise}%[\textbf{5 баллов} за каждый пункт]
Двое играют на деньги, одновременно называя одно из чисел 1 или 2, 
и потом считая сумму $S$. Если $S$ четная, то первый выигрывает у второго $S$ долларов,
если $S$ нечетная, то второй выигрывает у первого $S$ долларов. 
\begin{enumerate}
	\item Постройте платежную матрицу (матрицу полезностей) каждого из игроков. 
	Будет ли эта игра игрой с нулевой суммой? % Ответ обоснуйте.
	\item Будут ли в этой игре доминирующие стратегии? % Ответ поясните.
	\item Будут ли положения равновесия по Нэшу в чистых стратегиях? % Ответ обоснуйте.
	\item Предположим, что игроки следуют смешанным стратегиям 
	\(P^\top=\begin{pmatrix} 0.3 & 0.7\end{pmatrix}\) и \(Q^\top=\begin{pmatrix} 0.25 & 0.75\end{pmatrix}\).
	Вычислите ожидаемый выигрыш каждого из игроков.
	\item Найдите положения равновесия по Нэшу в смешанных стратегиях.
	% \item Дайте интерпретацию равновесных (по Нэшу) стратегий.
	\item Найдите ожидаемый выигрыш (полезность) каждого из игроков
	в положении равновесия по Нэшу в смешанных стратегиях.
\end{enumerate}
\end{exercise}

\begin{exercise}
Решите предыдущую задачу при условии, что игроки называю одно из 
следующих чисел: 1, 2 или 3.
\end{exercise}

\begin{exercise}
Рассмотрим антагонистическую игру с матрицей
\[
	\begin{pmatrix}
	-2 & 2 & -1 & 0 & 1 \\
	2 & 3 & 1 & 2 & 2 \\
	3 & -3 & 2 & 4 & 3 \\
	-2 & 1 & -2 & -1 & 0
	\end{pmatrix}
\]
\begin{enumerate}
	\item Найдите верхнюю и нижнюю цену игры.
	\item Существует ли равновесие в чистых стратегиях? Ответ поясните.
	\item Можно ли уменьшить размер платежной матрицы игры?
	% Как называется этот способ и в чем он состоит?
	% \item Решите данную задачу геометрическими способом задачи $2\times n$.
	\item Найдите равновесие Нэша путем сведения к задаче линейного
	программирования % (решить графически).
\end{enumerate}
\end{exercise}

\begin{exercise}
Примените к платежным матрицам операцию доминирования. Проведите анализ игры до доминирования и 
после операции доминирования. Найдите равновесие Нэша и цену игры
\begin{align*}
	a)&\;\begin{pmatrix} 2 & 1& 0 & -1 & 4 \\ 3 & 4 & 1 & 1 & 4 \\
	-1 & 0 & 2 & -4 & 1 \\ 2 & 2 & 3 & 1 & 3 \\ 4 & 5 & 3 & 1 & 2 \end{pmatrix} &
	b)&\; \begin{pmatrix} 1 & 4 & 0 & -3 & 2 \\ 3 & 3 & 2 & 4  & 1 \\ 2 & 5 & 1 & 2 & 3 \\ 
	3 & 4 & -1 & 0 & 2 \\ 2 & 2 & 1 & 1 & 0 \end{pmatrix} \\
	c)&\; \begin{pmatrix} 4 & 3 & 3 & 4 & 4 \\ 3 & -1 & -5 & 1 & 5 \\ 8 & 2 & -6 & 0 & -5 \\ 
	2 & 0 & 1 & 4 & 5 \\ 2 & 1 & 3 & 5 & 6 \\ 4 & 4 & 3 & 6 & 5 \end{pmatrix}
\end{align*}
\end{exercise}

\begin{exercise}
Рассмотрим антагонистическую игру с матрицей
\[
	\begin{pmatrix}
	2 & -1 \\ -2 & 1
	\end{pmatrix}
\]
\begin{enumerate}
	\item Рассмотрим  смешанные стратегии игроков 
	\begin{align*}
		P^\top&=\begin{pmatrix} 0.4 & 0.6 \end{pmatrix} &
		Q^\top&=\begin{pmatrix} 0.8 & 0.2 \end{pmatrix}
	\end{align*}
	Найдите ожидаемые выигрыши каждого из игроков
	\item Найдите положения равновесия по Нэшу и цену игры.
\end{enumerate}
\end{exercise}

\begin{exercise}
Найти равновесие (по Нэшу) в смешанных стратегиях и цену игры в
игре с нулевой суммой
\[
	\begin{pmatrix}
	-20 & 2 & 22 & -15 \\ 20 & -8 & -11 & 0
    \end{pmatrix}
\]
\end{exercise}

\begin{exercise}
Для антагонистической игры с матрицей
\begin{align*}
	a)&\;\begin{pmatrix} 4 & 2 & 1 & 5 \\ 2 & 3 & 6 & 3 \end{pmatrix} &
	b)&\; \begin{pmatrix} 2 & 4 \\ 0 & 5 \\ 2 & 6 \\ 3 & -4 \\ 1 & 5 \\ 3 & -1\end{pmatrix} &
	c)&\; \begin{pmatrix} -2 & 3 & 4 & 1 & 3 \\ 6 & -5 & 3 & 3 & -1 \end{pmatrix}
\end{align*}
найдите  равновесие Нэша.
\end{exercise}

\begin{exercise}
Игра <<вооружение помехи>>. Сторона $A$ располагает тремя видами вооружений
$A_1,A_2,A_3$, а сторона $B$ -- тремя видами помех $B_1,B_2,B_3$. Вероятность решения
боевой задачи стороной $A$ при различных видах вооружения и помех задана матрицей
\begin{center}
	\begin{tabular}{|c|c|c|c|}\hline
	& $B_1$ & $B_2$ & $B_3$ \\ \hline
	$A_1$ & 0.8 & 0.2 & 0.4 \\ \hline
	$A_2$ & 0.4 & 0.5 & 0.6 \\ \hline
	$A_3$ & 0.1 & 0.7 & 0.3 \\ \hline
	\end{tabular}
\end{center}
Сторона $A$ стремиться решить боевую задачу, сторона $B$ -- воспрепятствовать этому.
\begin{itemize}
	\item Найдите верхнюю и нижнюю цену игры. Будут ли в этой игре положения равновесия
	(по Нэшу) в чистых стратегиях?
\end{itemize}
Для удобства записи умножим матрицу на 10.
\begin{itemize}
	\item Напишите пару двойственных задач для нахождения равновесия в смешанных стратегиях.
\end{itemize}
Пусть известны оптимальные решения двойственных задач линейного программирования:
для игрока $A$
\begin{align*}
	x_1&=\frac{1}{32} & x_2&=\frac{3}{16} & x_3&=0
\end{align*}
для игрока $B$
\begin{align*}
	y_1&=\frac{3}{32} & y_2&=\frac{4}{32} & y_3&=0
\end{align*}
\begin{itemize}
	\item Найдите оптимальные стратегии каждого из игроков и цену игры.
\end{itemize}
\end{exercise}

\begin{exercise}
Полковник Блотто командует тремя отрядами. Перед ним три высоты.
Он должен решить, сколько отрядов послать на захват каждой высоты.
Его противник, граф Балони, также имеет в подчинении три отряда и должен принять такое же решение. 
Если на одной из высот у одного противника есть численное превосходство, то он захватывает эту высоту. 
Если нет, то высота остается нейтральной территорией. 
Выигрыш каждого игрока равен количеству захваченных им высот, 
минус количество высот, захваченных противником. 
Постройте матрицу игры и найдите положение равновесия по Нэшу в этой игре.
\end{exercise}


\subsection{Игры с ненулевой суммой}

\begin{exercise}
Рассмотрим платежную матрицу участников A и B некоторого парного турника,
которые придерживаются в нем одной их двух стратегий
%биматричную игру
\begin{center}
	\begin{tabular}{|c||c|c|}
	\hline
	% after \\: \hline or \cline{col1-col2} \cline{col3-col4} ...
	& $s_{-1}$ & $s_{-2}$  \\ \hline \hline
	$s_1$ & 1, 2 & 4, 3  \\ \hline
	$s_2$ & 3, 4 & 2, 3  \\ %\hline
	\hline
	\end{tabular}
\end{center}
\begin{enumerate}
	\item Рассмотрим  смешанные стратегии игроков 
	\begin{align*}
		P^\top&=\begin{pmatrix} 0.7 & 0.3 \end{pmatrix} &
		Q^\top&=\begin{pmatrix} 0.6 & 0.4 \end{pmatrix}
	\end{align*}
	Найдите ожидаемые выигрыши каждого из игроков
	\item Найдите положения равновесия по Нэшу в чистых и смешанных стратегиях
\end{enumerate}
\end{exercise}

\begin{exercise}
Рассмотрим биматричную игру
\begin{center}
	\begin{tabular}{|c||c|c|}
	\hline
	% after \\: \hline or \cline{col1-col2} \cline{col3-col4} ...
	& $s_{-1}$ & $s_{-2}$  \\ \hline \hline
	$s_1$ & 4, 2 & 2, 0  \\ \hline
	$s_2$ & 2, 2 & 3, 5  \\ %\hline
	\hline
	\end{tabular}
\end{center}
\begin{enumerate}
	\item Рассмотрим смешанные стратегии игроков 
	\begin{align*}
		P^\top&=\begin{pmatrix} 0.5 & 0.5 \end{pmatrix} &
		Q^\top&=\begin{pmatrix} 0.2 & 0.8 \end{pmatrix}
	\end{align*}
	Найдите ожидаемые выигрыши каждого из игроков
	\item Найдите положения равновесия по Нэшу в чистых и смешанных стратегиях
\end{enumerate}
\end{exercise}

\begin{exercise}
Рассмотрим биматричную игру
\begin{center}
	\begin{tabular}{|c||c|c|c|}
	\hline
	% after \\: \hline or \cline{col1-col2} \cline{col3-col4} ...
	& $s_{-1}$ & $s_{-2}$  & $s_{-3}$ \\ \hline \hline
	$s_1$ & 4, 1 & 2, 2 & 1, 3  \\ \hline
	$s_2$ & 2, 2 & 3, 5 & 0, 4 \\ %\hline
	\hline
	\end{tabular}
\end{center}
\begin{itemize}
	\item Рассмотрим смешанные стратегии игроков 
	\begin{align*}
		P^\top&=\begin{pmatrix} 0.6 & 0.4 \end{pmatrix} &
		Q^\top&=\begin{pmatrix} 0.2 & 0.3 & 0.5 \end{pmatrix}
	\end{align*}
	Найдите ожидаемые выигрыши каждого из игроков
	\item Найдите положения равновесия по Нэшу в чистых и смешанных стратегиях
\end{itemize}
\end{exercise}

\begin{exercise}
Рассмотрим биматричную игру
\begin{center}
	\begin{tabular}{|c||c|c|}
	\hline
	% after \\: \hline or \cline{col1-col2} \cline{col3-col4} ...
	& $s_{-1}$ & $s_{-2}$  \\ \hline \hline
	$s_1$ & 4, 2 & 2, 0  \\ \hline
	$s_2$ & 2, 2 & 3, 5  \\ \hline
	$s_3$ & 3, 1 & 2, 3 \\ 
	\hline
	\end{tabular}
\end{center}
\begin{enumerate}
	\item Рассмотрим смешанные стратегии игроков 
	\begin{align*}
		P^\top&=\begin{pmatrix} 0.4 & 0.4 & 0.2 \end{pmatrix} &
		Q^\top&=\begin{pmatrix} 0.3 & 0.7 \end{pmatrix}
	\end{align*}
	Найдите ожидаемые выигрыши каждого из игроков
	\item Найдите положения равновесия по Нэшу в чистых и смешанных стратегиях
\end{enumerate}
\end{exercise}


\begin{exercise}[Дуополия Курно]
Пусть $Q_i$ -- объем выпуска, $cQ_i$ --
издержки фирмы $i=1,2$. Функция спроса имеет вид ($a>c>0$)
\[
	P(Q)=\begin{cases}
	a-Q, & Q\leq a \\
	0, & Q>a
	\end{cases}
\]
Доход фирмы определяется равенством $(P(Q_1+Q_2)-c)Q_i$.

Каждая фирма имеет две возможности: <<мелкосерийное производство>>
$Q^l=(a-c)/4$ и <<крупносерийное производство>> $Q^h=(a-c)/3$.
\begin{enumerate}
	\item Напишите биматричную игру в нормальной форме.
	\item Найдите положения равновесия (по Нэшу) в чистых и смешанных
	стратегиях.
\end{enumerate}
\end{exercise}

\begin{exercise}[Дуополия Бертрана]
Пусть на рынке минеральной воды присутствуют две конкурирующие фирмы $A$ и $B$. 
Постоянные издержки каждой из них равны 300
(вне зависимости от объема продаж). Каждая фирма
должна выбрать либо <<высокую>> цену на свою продукцию $P_h=1$, 
либо <<низкую>> цену $P_l=0.5$ (цена за бутылку). При при <<высокой>> цене на 
рынке можно продать 1000 бутылок, при <<низкой>> цене -- 2000 бутылок.
Если компании выбирают одинаковую цену, то они делят объемы продаж поровну.
Если компании выбирают разные цены, то рынок полностью захватывает компания
с более низкой ценой (другая ничего не продает).
\begin{enumerate}
	\item Постройте платежную матрицу (матрицу полезностей) каждого из игроков. 
	Будет ли эта игра игрой с нулевой суммой? % Ответ обоснуйте.
	\item Будут ли в этой игре доминирующие стратегии? % Ответ поясните.
	\item Будут ли положения равновесия по Нэшу в чистых стратегиях? % Ответ обоснуйте.
	\item Найдите положения равновесия по Нэшу в смешанных стратегиях.
	\item Дайте интерпретацию равновесных (по Нэшу) стратегий.
	\item Найдите ожидаемый выигрыш (полезность) каждого из игроков
	в положении равновесия по Нэшу в смешанных стратегиях.
\end{enumerate}
\end{exercise}

\begin{exercise}[Дуополия Бертрана]
Пусть на рынке минеральной воды присутствуют две конкурирующие фирмы $A$ и $B$. 
Постоянные издержки каждой из них равны \$5000
(вне зависимости от объема продаж). Каждая фирма
должна выбрать либо <<высокую>> цену на свою продукцию $P_h=\$2$, либо <<низкую>> цену $P_l=\$1$
(цена за бутылку). Тогда:
\begin{enumerate}
	\item при <<высокой>> цене на рынке можно продать 5000 бутылок,
	\item при <<низкой>> цене на рынке можно продать 10000 бутылок,
	\item если компании выбирают одинаковую цену, то они делят объемы продаж поровну,
	\item если компании выбирают разные цены, то рынок полностью захватывает компания
	с более низкой ценой (другая ничего не продает).
\end{enumerate}
Постройте матрицу игры. Будет ли это игра с нулевой суммой? Найдите положения равновесия
по Нэшу и цену игры.
\end{exercise}

\begin{exercise}
Рассмотрим биматричную игру
\begin{center}
	\begin{tabular}{|c||c|c|c|c|}
	 \hline
	% after \\: \hline or \cline{col1-col2} \cline{col3-col4} ...
	& $s_{-1}$ & $s_{-2}$ & $s_{-3}$ & $s_{-4}$\\ \hline \hline
	$s_1$ & 2, 2 & 1, 1 & 1, 3 & 2, 1\\ \hline
	$s_2$ & 2, 4 & 1, 3 & 1, 2 & 4, 2\\ \hline
	$s_3$ & 3, 2 & 2, 3 & 2, 5 & 3, 1 \\ \hline
	$s_4$ & 4, 1 & 1, 4 & 3, 1 & 2, 2 \\
	\hline
	\end{tabular}
\end{center}
Проведите процедуру последовательного удаления доминируемых стратегий.
Будут ли в этой игре положения равновесия по Нэшу и по Парето в чистых стратегиях?
\end{exercise}

\begin{exercise}
У двух авиапассажиров, следовавших одним рейсом, пропали чемоданы. 
Авиакомпания готова возместить ущерб каждому пассажиру. 
Для того, чтобы определить размер компенсации, каждого пассажира просят сообщить, 
во сколько он оценивает содержимое своего чемодана. 
Каждый пассажир может назвать сумму не менее \$90 и не более \$100. 
Условия компенсации таковы: если оба сообщают одну и ту же сумму, 
то каждый получит эту сумму в качестве компенсации. 
Если же заявленный одним из пассажиров ущерб окажется меньше, 
чем заявленный ущерб другого пассажира, то каждый пассажир получит компенсацию, 
равную меньшей из заявленных сумм. При этом тот, кто заявил меньшую сумму,
получит дополнительно \$2, тот, кто заявил большую сумму — дополнительно потеряет два доллара.
Постройте матрицу игры и найдите положение равновесия по Нэшу в этой игре.
\end{exercise}

\begin{exercise}
Винни Пух и Пятачок решают, к кому они пойдут в гости --- к Пуху
(стратегия 1), Пятачку (стратегия 2) или к Кролику (стратегия 3).
Друзья могут пойти как вместе так и по раздельности. С точки
зрения Пуха, удовольствие от похода в гости оценивается так: 1, 2
и 4 соответственно. С точки зрения Пятачка удовольствия таковы: 3,
2 и 2 соответственно. Присутствие спутника (совместный поход в
гости) добавляет одну единицу удовольствия каждому. Найти
равновесие в этой игре.
\end{exercise}

\begin{exercise}
Две фирмы делят рынок. У каждой фирмы три стратегии --- выставить
низкую цену, среднюю цену и высокую цену $p_1=2$, $p_2=4$,
$p_3=7$. Предположим, что спрос на рынке составляет $100$ единиц
продукции, при равенстве цен он делится поровну между фирмами,
если одна фирма выставила высокую цену, а другая среднюю, то спрос
составляет $30$ и $70$ единиц соответственно. При средней и низкой
цене спрос также делится на $30$ и $70$, а при высокой и низкой
он составляет $10$ и $90$ единиц соответственно. Найти положение
равновесия по Нэшу.
\end{exercise}

\begin{exercise}
Две радиостанции делят рынок, выбирая из двух форматов вещания: новости и музыка.
Целевая аудитория каждого из форматов равна 38\% и 62\% соответственно. Если радиостанции
выбирают одинаковый формат вещания, то они делят целевую аудитория пополам.
Если Если радиостанции выбирают разный формат вещания, то каждая ``захватывает''
всю целевую аудиторию своего формата вещания. Выигрыш радиостанции -- её доля аудитории.
\begin{enumerate}
	\item Запишите игру в нормальной форме. 
	Будет ли эта игра игрой с нулевой суммой? % Ответ обоснуйте.
	\item Найдите положения равновесия по Нэшу в чистых и смешанных стратегиях
	и выигрыши игроков в равновесии.
\end{enumerate}
\end{exercise}

% \subsection{Байесовские игры}

% \begin{exercise}
% Двое лыжников едут, разогнавшись, по одной лыжне навстречу друг другу. 
% Каждый должен решить, уступать лыжню или нет. Никто не хочет уступать друг другу дорогу. 
% Однако если никто лыжню не уступит, то произойдет столкновение. 

% Первый лыжник считает, что второй лыжник с вероятностью 0.5 — принципиальный, 
% то есть получает большой моральный штраф, 
% если будет уступать лыжню. Тогда игра имеет вид:
% \begin{center}
% 	\begin{tabular}{c|c|c|}
% 	& Уступить 2 & Не уступить 2 \\ \hline
% 	Уступить 1 & $0; -4$ & $-1;1$ \\ \hline
% 	Не уступить 1& $1; -5$ & $-4; -4$ \\ \hline
% 	\end{tabular}
% \end{center}
% Также первый лыжник считает, что второй лыжник с вероятностью 0.5 — обычный добродушный лыжник. 
% Тогда игра имеет вид:
% \begin{center}
% 	\begin{tabular}{c|c|c|}
% 	& Уступить 2 & Не уступить 2 \\ \hline
% 	Уступить 1 & $0; 0$ & $-1;1$ \\ \hline
% 	Не уступить 1& $1;-1$ & $-4; -4$ \\ \hline
% 	\end{tabular}
% \end{center}
% Первый лыжник — добродушный. Найдите равновесия Байеса-Нэша
% \end{exercise}

% \begin{exercise}
% В деревне живет 11 лыжников, некоторые из которых имеют добродушный нрав, а некоторые — злобный. 
% Всем жителям известно, что добродушных лыжников в деревне 8 человек, а остальные лыжники злобные. 

% 1 января ближе к вечеру один из лыжников решил прокатиться по единственной в округе лыжне. 
% При подъезде к очередному глубокому оврагу он заметил на противоположной стороне другого лыжника, 
% однако из-за расстояния не смог его узнать. Каждый спортсмен независимо от другого решет, 
% съезжать ли ему в овраг или пропустить другого лыжника. 
% Если оба решат съехать, то произойдет столкновение и каждый получит платеж -4. 

% Добродушный лыжник получает моральное удовлетворение от пропуска другого лыжника в размере 1, 
% если другой воспользовался преимуществом, и ничего не получает, если другой преимуществом не воспользовался. 
% Если же пропустили его и он этим воспользовался, то он испытывает угрызения совести в размере -1. 

% Злобный лыжник испытывает неудовольствие от пропуска другого лыжника в размере -1, 
% если другой воспользовался преимуществом, и ничего не получает, 
% если другой преимуществом не воспользовался. Если же пропустили его и он этим воспользовался, 
% то он испытывает удовольствие в размере 1.

% Найдите равновесия Байеса-Нэша
% \end{exercise}

% \begin{exercise}
% Коля и Маша играют в упрощенную версию покера. У каждого игрока может быть на руках либо 
% хорошая комбинация карт, либо плохая. Оба игрока должны одновременно решить, 
% повышать ставку или нет. Коля по неосторожности <<засветил>> свои карты и теперь Маша знает, 
% что у Коли хорошая карта. Коля достоверно не знает, какие карты у Маши, но по его оценкам, 
% с вероятностью 0.7 у нее плохая карта, а с вероятностью 0.3 -- хорошая. 
% Каждый из игроков знает свою карту. Если у Маши плохая карта, то игра для Коли выглядит следующим образом:
% \begin{center}
% 	\begin{tabular}{c|c|c|}
% 	& Повышать 2 & Не повышать 2 \\ \hline
% 	Повышать 1 & $2; -2$ & $1;-1$ \\ \hline
% 	Не повышать 1& $-1; 1$ & $1; -1$ \\ \hline
% 	\end{tabular}
% \end{center}
% Если у Маши хорошая карта, то игра имеет вид:
% \begin{center}
% 	\begin{tabular}{c|c|c|}
% 	& Повышать 2 & Не повышать 2 \\ \hline
% 	Повышать 1 & $0; 0$ & $1;-1$ \\ \hline
% 	Не повышать 1& $-1; 1$ & $0; 0$ \\ \hline
% 	\end{tabular}
% \end{center}
% Найдите все равновесия Байеса-Нэша в этой игре.
% \end{exercise}

% \begin{exercise}
% После очередного сражения командующий войсками страны А предлагает командующему войсками 
% страны В капитулировать. О численности войск и состоянии армии противника после боя командирам неизвестно.
% Командующий войсками страны В, по численности армии, изначально превосходящей страну А, 
% предполагает, что с вероятностью $0.7$ командующий войсками страны А может блефовать и 
% сам находится в невыгодном положении. Следующий бой скоро и командирам нужно решить, что делать -- 
% продолжить сражение или капитулировать. При этом, до командующего войсками страны А доходили слухи, 
% что командующий войсками страны В -- темный маг, который может призвать нечисть и выиграть битву. 
% Так как это, по его мнению, невозможно, то он относится к такой информации скептически и верит ей лишь на 15\%. 
% Независимо от того, в каком положении находится командир А, если он капитулирует, то получает $-100$ (минус 100), 
% если противник, кем бы он ни был, собирался атаковать, и $-300$ (минус 300), если собирался сдаться. 
% Если командир А, находясь в проигрышном положении, идет в бой с готовым атаковать противником, 
% то теряет армию и сам умирает, следовательно получает $-500$ (минус 500), если бой был против темного мага и $-800$
% (минус 800),  если против обычного человека. Все немного лучше, если он находится в выигрышном положении -- тогда, 
% если в бою его атакует обычный противник, то оба в таком исходе получают 0 т.к. сражение произошло в ничью; 
% если же войска противника вел темный маг, то командир А получает также $-500$ (минус 500). 
% Если командир А в выигрышном положении нападает получает 500, если противник сдался, будучи темным магом, и 300, 
% если противник темным магом не был. Если его положение проигрышное, то к этим платежам прибавляется +200. 
% Для капитана В, если он темный маг, при вступлении в бой все нипочем, он получает 700, но если по каким-то причинам 
% он капитулирует, то его платеж равен $-1000$ (минус 1000). Если В -- обычный человек, и идет в бой против противника в
% проигрышном положении, то получает 100, если тот дает бой, и 500, если капитулирует; если противник в выигрышном
% положении, то 0 (как уже было сказано) и 300 соответственно. Если обычный капитан В решает капитулировать, то
% получает $-100$ (минус 100), если противник в проигрышном положении идет в бой, и $-300$ (минус 300), 
% если капитулирует. К этим платежам прибавляется по 100, если противник в выигрышном положении.
% \begin{enumerate}
% 	\item Формализуйте игру
% 	\item Напишите, как игра выглядит с точки зрения каждого из участников.
% 	\item Найдите все равновесия Байеса-Нэша в данной игре.
% \end{enumerate}
% \end{exercise}