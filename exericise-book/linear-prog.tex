% !TEX root = exercise-book-opt.tex

\begin{exercise}
Рассмотрим задачи линейного программирования
\begin{align*}
	& \begin{gathered}
		\max(3x+5y) \\
		s.t.\left\{\begin{aligned}
			x+y &\leqslant5 \\ 2x+y &\leqslant8 \\ x,y&\geqslant0
		\end{aligned}\right.
	\end{gathered} &
	& \begin{gathered}
		\max(7x+4y) \\
		s.t.\left\{\begin{aligned}
			2x+5y &\leqslant30 \\ 2x+y &\leqslant14 \\ x,y&\geqslant0
		\end{aligned}\right.
	\end{gathered} \\
	& \begin{gathered}
		\max(4x+5y) \\
		s.t.\left\{\begin{aligned}
			x+3y &\leqslant15 \\ 4x+3y &\leqslant24 \\ x,y&\geqslant0
		\end{aligned}\right.
	\end{gathered} &
	& \begin{gathered}
		\max(8x+3y) \\
		s.t.\left\{\begin{aligned}
			2x+5y &\leqslant35 \\ 5x+3y &\leqslant40 \\ x,y&\geqslant0
		\end{aligned}\right.
	\end{gathered}
\end{align*}
\begin{enumerate} 
	\item Решите графическии (прямую) задачу
	\item Напишите и решите графически двойственную задачу.
\end{enumerate}
\end{exercise}

\begin{exercise}
Решите графически следующие задачи оптимизации
\begin{align*}
	& \begin{gathered}
		\max(5x+4y) \\
		s.t.\left\{\begin{aligned}
			x+3y &\leqslant18 \\ x+2y &\leqslant13 \\ 
			3x+2y &\leqslant27 \\ x,y&\geqslant0
		\end{aligned}\right.
	\end{gathered} &
	& \begin{gathered}
		\max(4x+3y) \\
		s.t.\left\{\begin{aligned}
			x+4y &\leqslant 28 \\ 2x+3y &\leqslant 26 \\ 
			x+y &\leqslant 11 \\ 2x+y &\leqslant 20 \\ x,y&\geqslant0
		\end{aligned}\right.
	\end{gathered} \\
	& \begin{gathered}
		\min(7x+6y) \\
		s.t.\left\{\begin{aligned}
			3x+y &\geqslant 9 \\ 4x+3y &\geqslant 22 \\ 
			x+3y &\leqslant 10 \\ x,y&\geqslant0
		\end{aligned}\right.
	\end{gathered} &
	& \begin{gathered}
		\min(2x+5y) \\
		s.t.\left\{\begin{aligned}
			3x+y &\geqslant 10 \\ 2x+y &\geqslant 8 \\ 
			x+3y &\geqslant 9 \\ x+6y &\geqslant 12 \\ x,y&\geqslant0
		\end{aligned}\right.
	\end{gathered}
\end{align*}
\end{exercise}

\begin{exercise}
Решите графически следующие задачи оптимизации
\begin{align*}
	& \begin{gathered}
		\max(5x-4y) \\
		s.t.\left\{\begin{aligned}
			5x+2y &\leqslant 16 \\ 2x-7y &\leqslant 22 \\ 
			-5x-y &\leqslant 19 \\ -x+3y &\leqslant 7
		\end{aligned}\right.
	\end{gathered} &
	& \begin{gathered}
		\max(-3x+2y) \\
		s.t.\left\{\begin{aligned}
			x+2y &\leqslant 9 \\ 3x-y &\leqslant 13 \\ 
			-2x-7y &\leqslant 22 \\ -5x+y &\leqslant 18 \\ 
			-x+4y &\leqslant 15
		\end{aligned}\right.
	\end{gathered} 
\end{align*}
\end{exercise}

% \begin{exercise}
% Рассмотрим задачу линейного программирование в матричном виде 
% \begin{gather*}
% 	\max(\vectf^\top\vectx) \\
% 	s.t.\left\{\begin{aligned}
% 		\matrixA\vectx & \leq\vectc \\ \vectx&\geq0
% 	\end{aligned}\right.
% \end{gather*}
% Для каждого из примеров решите 
% \end{exercise}

% \begin{exercise}
% Найдите решение задачи оптимизации
% \begin{gather*}
% 	\min (3x_1+4x_2) \\ 
% 	\left\{\begin{aligned} 
% 	& 3x_1+2x_2\geq 13 \\ & 5x_1+x_2\geq 10 \\ 
% 	& x_1+2x_2\geq 7 \\ & x_1,x_2\geq0
% 	\end{aligned}\right.
% \end{gather*}
% \end{exercise}

% \begin{exercise}
% Решите задачу оптимизации
% \begin{gather*}
% 	\max (3x_1+4x_2+2x_3+x_4)  \\
% 	\left\{\begin{aligned}
% 	2x_1+x_2+5x_3+5x_4 &\leq  40 \\
% 	x_1+2x_2+3x_3+2x_4 &\leq 30 \\
% 	x_1,x_2,x_3,x_4 & \geq 0
% 	\end{aligned}\right.
% \end{gather*}
% \begin{enumerate}
% 	\item симплекс-методом
% 	\item через двойственную задачу
% \end{enumerate}
% \end{exercise}

% \begin{exercise}
% Решите задачу оптимизации
% \begin{gather*}
% 	\max(x_1+2x_2+2x_3)\\
% 	\left\{\begin{aligned}
% 	   3x_1+x_2+x_3 &\leq 20 \\
% 		x_1+2x_2+2x_3 &\leq30 \\
% 		x_1, x_2, x_3 &\geq0
% 	\end{aligned}\right.
% \end{gather*}
% \begin{enumerate}
% 	\item симплекс-методом
% 	\item через двойственную задачу
% \end{enumerate}
% \end{exercise}

% \begin{exercise}
% Решите задачу оптимизации с использованием симплекс-метода
% \begin{gather*}
% 	\max(x_1+2x_2+2x_3+4x_4)\\
% 	\left\{\begin{aligned}
% 	   2x_1+2x_2+x_3+4x_4 &\leq 30 \\ 
% 		x_1+2x_2+2x_3+2x_4 &\leq 50 \\ 
% 		x_1, x_2, x_3, x_4 &\geq0
% 	 \end{aligned}\right.
% \end{gather*}
% \begin{enumerate}
% 	\item симплекс-методом
% 	\item через двойственную задачу
% \end{enumerate}
% \end{exercise}

% \begin{exercise}
% Решите задачу оптимизации
% \begin{gather*}
% 	\max(2x_1+2x_2+x_3+x_4)\\
% 	\left\{\begin{aligned}
% 		2x_1+2x_2+x_3+x_4 &\leq 20 \\ 
% 		x_1+x_2+2x_3+2x_4 &\leq 30 \\ 
% 		x_1, x_2, x_3, x_4 &\geq0
% 	\end{aligned}\right.
% \end{gather*}
% \begin{enumerate}
% 	\item симплекс-методом
% 	\item через двойственную задачу
% \end{enumerate}
% \end{exercise}

\begin{exercise}
Фирма производит четыре товара и использует для производства два ресурса.
Норма затрат ресурсов, количество ресурсов и прибыль от каждой единицы
товара приведены в таблице
\begin{center}
\begin{tabular}{|c|c|c|c|c||c|}
	\hline
	& Товар 1 & Товар 2 & Товар 3 & Товар 4 &  Количество \\
	& & & & &  ресурса \\
	\hline
	Ресурс 1 & 4 & 4 & 1 & 2 & 100\\ \hline
	Ресурс 2 & 5 & 3 & 2 & 1 & 150 \\ \hline \hline
	Цена & 20 & 12 & 4 & 2 &  \\ \hline
\end{tabular}
\end{center}
Предполагается, что нормы затрат постоянны и цены постоянны.

Постройте модель оптимизации производства и решите её численно 
(MS Excel/Python).
\end{exercise}

\begin{exercise}
Фирма производит три товара и использует для производства два ресурса.
Норма затрат ресурсов, количество ресурсов и прибыль от каждой единицы 
товара приведены в таблице
\begin{center}
\begin{tabular}{|c|c|c|c||c|}
	\hline 
	& Товар 1 & Товар 2 & Товар 3 & Количество \\
	& & & & ресурса \\
	\hline
	Ресурс 1 & 2 & 1 & 5 & 100 \\ \hline
	Ресурс 2 & 4 & 2 & 3 & 120 \\ \hline \hline
	Цена & 3 & 8 & 2 & \\ \hline
\end{tabular}
\end{center}
Предполагается, что нормы затрат постоянны и цены постоянны.

Постройте модель оптимизации производства и решите её численно 
(MS Excel/Python).
\end{exercise}

\begin{exercise}
Фирма производит три товара и использует для производства два ресурса.
Норма затрат ресурсов, количество ресурсов и прибыль от каждой единицы
товара приведены в таблице
\begin{center}
\begin{tabular}{|c|c|c|c||c|}
	\hline
	& Товар 1 & Товар 2 & Товар 3 & Количество  \\
	& & & & ресурса \\
	\hline
	Ресурс 1 & 2 & 1 & 5 & 100 \\ \hline
	Ресурс 2 & 5 & 2 & 5 & 220 \\ \hline \hline
	Цена & 3 & 8 & 2 & \\ \hline
\end{tabular}
\end{center}
Предполагается, что нормы затрат постоянны и цены постоянны.

Постройте модель оптимизации производства и решите её численно 
(MS Excel/Python).
\end{exercise}
	

% \begin{exercise}
% Фирма производит три товара и использует для производства два ресурса.
% По плану первого товара нужно произвести не менее 100 единиц, второго --
% не менее 120, третьего -- не менее 150 ед.
% Норма затрат ресурсов и цена на ресурсы приведены в таблице
% \begin{center}
%  \begin{tabular}{|c|c|c||c|}
%   \hline 
%   & Ресурс 1 & Ресурс 2 & План \\
%   \hline
%   Товар 1 & 2 & 1 &  100 \\ \hline
%   Товар 2 & 4 & 3 &  120 \\ \hline
%   Товар 3 & 3 & 5 &  150 \\ \hline \hline
%   Цена & 3 & 6 & \\ \hline
%  \end{tabular}
% \end{center}
% Предполагается, что нормы затрат постоянны и цены экзогенны.
% \begin{enumerate}[i)]
%  \item Постройте модель оптимизации затрат ресурсов.
%  \item Постройте двойственную задачу.
%  \item Найдите оптимального количество ресурсов.
%  \item Найдите решение двойственной задачи и дайте
%  экономическую интерпретацию этого решения.
% \end{enumerate}
% \end{exercise}

\begin{exercise}%[\textbf{11 баллов}]
Фирма <<Московия>> заключила контракт с компанией АЛРОСА на покупку
промышленного золота для его реализации в пяти городах в объеме:
Самара -- 80 кг, Москва -- 260 кг, Ростов-на-Дону -- 100 кг,
Санкт-Петербург -- 140 кг, Нижний Новгород -- 120 кг. Компания
АЛРОСА располагает тремя месторождениями: <<Мирное>>, <<Удачный>> и
<<Полевое>>, которые планируют за год выработать соответственно
200, 250 и 250 кг золота.

Постройте модель оптимизации фрахта специализированного транспорта,
обеспечивающего полное удовлетворение заявок покупателя, при
заданной системе тарифов (на 1 кг)
\begin{center}\small
	\begin{tabular}{|c|c|c|c|c|c|}
	\hline
	% after \\: \hline or \cline{col1-col2} \cline{col3-col4} ...
	& Самара & Москва & Ростов-на-Дону & С.-Пб. & Н. Новгород \\ \hline
	<<Мирное>> & 7 & 9 & 15 & 4 & 18\\
	<<Удачный>> & 13 & 25 & 8 & 15 & 5 \\
	<<Полевое>> & 5 & 11 & 6 & 20 & 12\\
	\hline
	\end{tabular}
\end{center}
Найдите \textbf{численно} оптимальное решение.
\end{exercise}

\begin{exercise}
Московский филиал <<The Coca-Cola Company>>, выпускающей напитки
приблизительного равного спроса (Sprite, Coca-Cola, Fanta),
складируемых в разных местах, должен поставить свою продукцию в
четыре крупных супермаркета: <<Ашан>>, <<Карусель>>, <<Седьмой
Континент>> и <<Арбатский>>. Каждая упаковка содержит 12 банок
емкостью 0.33 литра. Тарифы на доставку, объемы запасов и заказы на
продукцию приведены в таблице.
\begin{center}\footnotesize
	\begin{tabular}{|c|c|c|c|c|c|}
	\hline
	% after \\: \hline or \cline{col1-col2} \cline{col3-col4} ...
	& \multicolumn{4}{|c|}{Супермаркеты} & \\ \hline
	Склады & Ашан & Карусель & Перекрёсток& Дикси &
	Запасы, уп.
	\\ \hline
	Coca-Cola & 6 & 4 & 9 & 5 & 400 \\
	Sprite & 5 & 7 & 8 & 6 & 300 \\
	Fanta & 9 & 4 & 6 & 7 & 200 \\ \hline
	Заказы, уп. & 150 & 250 & 150 & 350 &  \\
	\hline
	\end{tabular}
\end{center}
Постройте оптимизационную модель плана поставок напитков в
супермаркеты. Найдите \textbf{численно} оптимальное решение.
\end{exercise}

\begin{exercise}
Коммерческое предприятие реализует три группы товаров A, B и C.
Плановые нормативы затрат ресурсов (на 1 тыс рублей товарооборота),
доход от продажи товаров (на 1 тыс. рублей товарооборота)
приведены в таблице
\begin{center}\small
	\begin{tabular}{|l|c|c|c|c|}
	\hline
	& \multicolumn{3}{|c|}{Нормы затрат} & \\ \hline
	% after \\: \hline or \cline{col1-col2} \cline{col3-col4} ...
	Ресурсы & A & B & C & Объем ресурсов \\ \hline
	Рабочее время продавцов & 0.1 & 3 & 0.4 & 1100 \\
	Площадь торговых залов & 0.05 & 0.2 & 0.02 & 120 \\
	Площадь складских помещений & 3 & 0.02 & 2 & 8000 \\ \hline
	Доход & 3 & 1 & 4 &  \\
	\hline
	\end{tabular}
\end{center}
Постройте модель оптимизации для получения максимального дохода.
Найдите \textbf{численно} оптимальное решение.
\end{exercise}

% \begin{exercise}
% Для поддержания нормальной жизнедеятельности человеку ежедневно
% необходимо потреблять 118г белков, 56г жиров, 500г углеводов, 8г
% минеральных солей. Количество питательных веществ, содержащихся в
% 1кг имеющихся в магазине продуктов питания, а также их стоимость
% приведены в таблице
% \begin{center}
% {\small
% \begin{tabular}{|l|c|c|c|c|c|c|c|c|}
%   \hline
%    & \multicolumn{7}{|c|}{Содержание в 1 кг продуктов} &  \\ \hline
%   % after \\: \hline or \cline{col1-col2} \cline{col3-col4} ...
%    & мясо & рыба & молоко & масло & сыр & крупа & картофель & Нормы
%    \\ \hline
%   Белки, г & 180 & 190 & 30 & 70 & 260 & 130 & 21 & 118 \\
%   Жиры, г & 20 & 3 & 40 & 865 & 310 & 30 & 2 & 56 \\
%   Углеводы, г & 0 & 0 & 50 & 6 & 20 & 650 & 200 & 500 \\
%   Мин. соли, г & 9 & 10 & 7 & 12 & 60 & 20 & 70 & 8 \\ \hline
%   Стоимость, кг & 1.9 & 1.0 & 0.28 & 3.4 & 2.9 & 0.56 & 0.1 &  \\
%   \hline
% \end{tabular}
% }
% \end{center}
% Требуется составить модель оптимизации суточного рациона,
% содержащего не менее суточной потребности человека в белках, жирах,
% углеводах, минеральных солях и обеспечивающего минимальную стоимость
% продуктов.
% \end{exercise}

\begin{exercise}
Три нефтеперерабатывающих завода с (ежедневной) производительностью
6, 5 и 8 млн.т бензина снабжают три бензохранилища, (ежедневно)
потребность которых составляет  4, 8 и 7 млн. т бензина соответственно.
Бензин транспортируется в бензохранилища по бензопроводу. Стоимость
транспортировки составляет 0.3 руб за 1000 т на один км длины бензопровода.
В таблице приведены расстояния в км между заводами и хранилищами.
Отметим, что первый нефтеперерабатывающий завод не связан бензопроводом
с третьим бензохранилищем.
\begin{center}%\small
	\begin{tabular}{|c|c|c|c|c|} \hline
	& \multicolumn{3}{|c|}{Хранилища} & \\ \hline
 	Завод & 1 & 2 & 3 & Объем \\ \hline
 	1 & 120 & 180 & --- & 6 \\ \hline
	2 & 300 & 100 & 80 & 5 \\ \hline
	3 & 200 & 250 & 120 & 8 \\ \hline
	Вместимость & 4& 8 & 7 & \\ %\hline
	хранилища & & & & \\ \hline
	\end{tabular}
\end{center}
Постройте оптимизационную модель транспортировки бензина.
Найдите \textbf{численно} оптимальное решение.
\end{exercise}