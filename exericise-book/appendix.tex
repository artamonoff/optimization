% !TEX root = exercise-book-opt.tex

\subsection{Симметричные матрицы}

Пусть \(\matrixA\) -- (\(n\times n\)) симметричная матрица.

\begin{teorema}[Критерий Сильвестра]\label{SylvesterCiterion}
Пусть $\matrixA$ -- симметричная матрица и
$\Delta_1,\ldots,\Delta_n$ последовательность ее угловых миноров:
\begin{align*}
	\Delta_1&=a_{11} & \Delta_2&=\det\begin{pmatrix} a_{11} & a_{12} \\ a_{21} & a_{22} \end{pmatrix} & 
	&\ldots & \Delta_n&=\det\matrixA
\end{align*}
Тогда
\begin{enumerate}
	\item $\matrixA>0\iff\Delta_i>0$, $i=1,\ldots,n$.
	\item $\matrixA<0\iff(-1)^i\Delta_i>0$, $i=1,\ldots,n$.
	\item если знаки миноров не удовлетворяют предыдущим пунктам, 
	то матрица не знакоопределена
\end{enumerate}  
\end{teorema}

\begin{propuesta}\label{2times2definitness}
Пусть $\matrixA$ -- симметричная $2\times 2$ 
Тогда
\begin{align*}
	\matrixA\geq0 &\iff \begin{matrix} a_{11} \\ a_{22} \end{matrix} \geq0,\;\det\matrixA\geq0\\
	\matrixA\leq0  &\iff \begin{matrix} a_{11} \\ a_{22} \end{matrix} \leq0,\;\det\matrixA\geq0.
\end{align*}    
\end{propuesta}

Для  $3\times3$ матрицы обозначим центральные миноры 
\begin{align*}
	\Minor_{(12)}&=\det\begin{pmatrix} a_{11} & a_{12} \\ a_{21} & a_{22} \end{pmatrix} \\
	\Minor_{(13)}&=\det\begin{pmatrix} a_{11} & a_{13} \\ a_{31} & a_{33} \end{pmatrix} \\
	\Minor_{(23)}&=\det\begin{pmatrix} a_{22} & a_{23} \\ a_{32} & a_{33} \end{pmatrix} 
\end{align*}

\begin{propuesta}\label{3times3definitness}
Пусть $\matrixA$ -- симметричная $3\times 3$. Тогда
\begin{align*}
	\matrixA\geq 0 &\iff \begin{matrix} a_{11} \\ a_{22} \\ a_{33} \end{matrix}\geq0,\;
	\begin{matrix} \Minor_{(12)} \\ \Minor_{(13)} \\ \Minor_{(23)} \end{matrix}\geq0,\;
	\det\matrixA\geq0\\
	\matrixA\leq 0 &\iff \begin{matrix} a_{11} \\ a_{22} \\ a_{33} \end{matrix}\leq0,\;
	\begin{matrix} \Minor_{(12)} \\ \Minor_{(13)} \\ \Minor_{(23)} \end{matrix}\geq0,\;
	\det\matrixA\leq0.
\end{align*}
\end{propuesta}

\subsection{Выпуклые функции}

Пусть числовая функция \(f\) определена на % выпуклом множестве 
\(\Domain(f)\subset \R^n\)

\begin{teorema}
Дважды непрерывно дифференцируемая функция $f$ выпукла $\iff$ 
$\Hessian_f(\vectx)\geq0$ %(как симметричная матрица) 
для всех $\vectx\in\Domain(f)$.

Если $\Hessian_f(\vectx)>0$ для всех $\vectx\in\Domain(f)$, 
то функция строго выпукла на $\Domain(f)$.
\end{teorema}

\begin{col}
Дважды непрерывно дифференцируемая функция $f$ вогнута $\iff$ 
$\Hessian_f(\vectx)\leq0$ % (как симметричная матрица) 
для всех $\vectx\in\Domain(f)$.

Если $\Hessian_f(\vectx)<0$ для всех $\vectx\in\Domain(f)$, 
то функция строго вогнута
на $\Domain(f)$.
\end{col}

\begin{remark}
Знак гессиана проверяем по критерию Сильвестра 
\ref{SylvesterCiterion} или
используем Предложения \ref{2times2definitness}, \ref{3times3definitness}
\end{remark}
	