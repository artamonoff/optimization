% !TEX root = exercise-book-opt.tex

\subsection{Симметричные матрицы}

Пусть \(\matrixA\) -- (\(n\times n\)) симметричная матрица.

\begin{teorema}[Критерий Сильвестра]\label{SylvesterCiterion}
Пусть $\matrixA$ -- симметричная матрица и
$\Delta_1,\ldots,\Delta_n$ последовательность ее угловых миноров:
\begin{align*}
	\Delta_1&=a_{11} & \Delta_2&=\det\begin{pmatrix} a_{11} & a_{12} \\ a_{21} & a_{22} \end{pmatrix} & 
	&\ldots & \Delta_n&=\det\matrixA
\end{align*}
Тогда
\begin{enumerate}
	\item $\matrixA>0\iff\Delta_i>0$, $i=1,\ldots,n$.
	\item $\matrixA<0\iff(-1)^i\Delta_i>0$, $i=1,\ldots,n$.
	\item если знаки миноров не удовлетворяют предыдущим пунктам, 
	то матрица не знакоопределена
\end{enumerate}  
\end{teorema}

\begin{propuesta}\label{2times2definitness}
Пусть $\matrixA$ -- симметричная $2\times 2$ 
Тогда
\begin{align*}
	\matrixA\geq0 &\iff \begin{matrix} a_{11} \\ a_{22} \end{matrix} \geq0,\;\det\matrixA\geq0\\
	\matrixA\leq0  &\iff \begin{matrix} a_{11} \\ a_{22} \end{matrix} \leq0,\;\det\matrixA\geq0.
\end{align*}    
\end{propuesta}

Для  $3\times3$ матрицы обозначим центральные миноры 
\begin{align*}
	\Minor_{(12)}&=\det\begin{pmatrix} a_{11} & a_{12} \\ a_{21} & a_{22} \end{pmatrix} \\
	\Minor_{(13)}&=\det\begin{pmatrix} a_{11} & a_{13} \\ a_{31} & a_{33} \end{pmatrix} \\
	\Minor_{(23)}&=\det\begin{pmatrix} a_{22} & a_{23} \\ a_{32} & a_{33} \end{pmatrix} 
\end{align*}

\begin{propuesta}\label{3times3definitness}
Пусть $\matrixA$ -- симметричная $3\times 3$. Тогда
\begin{align*}
	\matrixA\geq 0 &\iff \begin{matrix} a_{11} \\ a_{22} \\ a_{33} \end{matrix}\geq0,\;
	\begin{matrix} \Minor_{(12)} \\ \Minor_{(13)} \\ \Minor_{(23)} \end{matrix}\geq0,\;
	\det\matrixA\geq0\\
	\matrixA\leq 0 &\iff \begin{matrix} a_{11} \\ a_{22} \\ a_{33} \end{matrix}\leq0,\;
	\begin{matrix} \Minor_{(12)} \\ \Minor_{(13)} \\ \Minor_{(23)} \end{matrix}\geq0,\;
	\det\matrixA\leq0.
\end{align*}
\end{propuesta}

\subsection{Выпуклые функции}

Пусть числовая функция \(f\) определена на % выпуклом множестве 
\(\Domain(f)\subset \R^n\)

\begin{teorema}
Дважды непрерывно дифференцируемая функция $f$ выпукла $\iff$ 
$\Hessian_f(\vectx)\geq0$ %(как симметричная матрица) 
для всех $\vectx\in\Domain(f)$.

Если $\Hessian_f(\vectx)>0$ для всех $\vectx\in\Domain(f)$, 
то функция строго выпукла на $\Domain(f)$.
\end{teorema}

\begin{col}
Дважды непрерывно дифференцируемая функция $f$ вогнута $\iff$ 
$\Hessian_f(\vectx)\leq0$ % (как симметричная матрица) 
для всех $\vectx\in\Domain(f)$.

Если $\Hessian_f(\vectx)<0$ для всех $\vectx\in\Domain(f)$, 
то функция строго вогнута
на $\Domain(f)$.
\end{col}

\begin{remark}
Знак гессиана проверяем по критерию Сильвестра 
\ref{SylvesterCiterion} или
используем Предложения \ref{2times2definitness}, \ref{3times3definitness}
\end{remark}


\subsection{Функция Лагранжа для ограничений равенства}

Рассмотрим задачи оптимизации с ограничениями равенства

\begin{align*}
	& \begin{gathered}
		\max f(\vectx) \\
		s.t.\left\{\begin{aligned}
			g_1(\vectx)&=c_1 \\ &\vdots \\ g_k(\vectx)&=c_k
		\end{aligned}\right.
	\end{gathered} &
	& \begin{gathered}
		\min f(\vectx) \\
		s.t.\left\{\begin{aligned}
			g_1(\vectx)&=c_1 \\ &\vdots \\ g_k(\vectx)&=c_k
		\end{aligned}\right.
	\end{gathered}
\end{align*}
Функция Лагранжа для этих задач
\[
	\Lagrange(\vectx,\vectlambda)=f(\vectx)-\sum_{j=1}^k \lambda_jg_j(\vectx)
\]
\textbf{Необходимые условия} (локального) условного экстремума
\[
	\left\{\begin{aligned}
		\Lagrange'_{x_i}&=0 & i&=1,\ldots,n\\
		g_j(\vectx) &= c_j & j&=1,\ldots,k
	\end{aligned}\right.
\]
Гессиан для функции Лагранжа (симметричная матрица)
\[
	\underset{(n+k)\times(n+k)}{\BordHessian_\Lagrange}=\begin{pmatrix}
		\frac{\partial^2 \Lagrange}{\partial x_i\partial x_j} & | &\frac{\partial^2 \Lagrange}{\partial x_i\partial \lambda_l} \\
		-- & + & -- \\
		\frac{\partial^2 \Lagrange}{\partial \lambda_s\partial x_j} & | &
		\frac{\partial^2 \Lagrange}{\partial \lambda_s\partial \lambda_l}
   \end{pmatrix}
\]
Из определения функции Лагранжа
\begin{itemize}
	\item \(\frac{\partial^2 \Lagrange}{\partial \lambda_s\partial \lambda_l}=0\)
	\item \(\frac{\partial^2 \Lagrange}{\partial \lambda_s\partial x_j}=-\frac{\partial g_s}{\partial x_j}\)
\end{itemize}
Явный вид гессиана
\begin{equation}\label{BorderedHessian}
	\underset{(n+k)\times(n+k)}{\BordHessian_\Lagrange}=
%    \begin{pmatrix}
%      \frac{\partial^2 \Lagrange}{\partial x_i\partial x_j} & | &\frac{\partial^2 \Lagrange}{\partial x_i\partial \lambda_l} \\
%      -- & + & -- \\
%      \frac{\partial^2 \Lagrange}{\partial \lambda_s\partial x_j} & | &
%      \frac{\partial^2 \Lagrange}{\partial \lambda_s\partial \lambda_l}
%    \end{pmatrix}=\\
	\begin{pmatrix} 
		\frac{\partial^2 \Lagrange}{\partial x_1\partial x_1} &  \cdots &
		\frac{\partial^2 \Lagrange}{\partial x_1\partial x_n} & -\frac{\partial g_1}{\partial x_1} & \cdots & 
		-\frac{\partial g_k}{\partial x_1} \\ 
		\frac{\partial^2 \Lagrange}{\partial x_2\partial x_1} &  \cdots &
		\frac{\partial^2 \Lagrange}{\partial x_2\partial x_n} & -\frac{\partial g_1}{\partial x_2} & \cdots & 
		-\frac{\partial g_k}{\partial x_2}\\
		\vdots & \ddots & \vdots & \vdots & \ddots & \vdots \\
		\frac{\partial^2 \Lagrange}{\partial x_n\partial x_1} & \cdots &
		\frac{\partial^2 \Lagrange}{\partial x_n\partial x_n}& -\frac{\partial g_1}{\partial x_n} & \cdots & 
		-\frac{\partial g_k}{\partial x_n}\\
		-\frac{\partial g_1}{\partial x_1} & \cdots &
		-\frac{\partial g_1}{\partial x_n} & 0 & \cdots & 0\\
		\vdots & \ddots & \vdots & \vdots & \ddots & \vdots\\
		-\frac{\partial g_k}{\partial x_1} &  \cdots &
		-\frac{\partial g_k}{\partial x_n} & 0 & \cdots & 0\\
	\end{pmatrix}
\end{equation}
Пусть $\Minor_{i}$ ($i=1,...,n+k$) -- главный минор матрицы $\BordHessian_\Lagrange$, 
образованный строками и столбцами с индексами $i,i+1,...,n+k$.

\begin{teorema}[Достаточные условия минимума]\label{EqualityConstraintMinSufficientCondition}
Пусть в точке $\hat{\vectx}$  ранг матрицы $(\frac{\partial g_j}{\partial x_i})$ максимален и 
эта точка удовлетворяет необходимым условия экстремума.\\
Тогда достаточным условием локального минимума является выполнение неравенств
\begin{equation}\label{EqualityConstraintMinSignSufficientCondition}
	(-1)^k\Minor_1(\hat{\vectx}),\ldots,(-1)^k\Minor_{n-k}(\hat{\vectx})>0.
\end{equation}
\end{teorema}
\begin{remark}
Условие \eqref{EqualityConstraintMinSignSufficientCondition} означает, что
все миноры $\Minor_1,\ldots,\Minor_{n-k}$ имеют знак $(-1)^k$.
\end{remark}

\begin{teorema}[Достаточные условия максимума]\label{EqualityConstraintMaxSufficientCondition}
Пусть в точке $\hat{\vectx}$  ранг матрицы $(\frac{\partial g_j}{\partial x_i})$ максимален и 
эта точка удовлетворяет необходимым условия экстремума.\\
Тогда достаточным условием наличия максимума является выполнение неравенств
\begin{align}\label{EqualityConstraintMaxSignSufficientCondition}
	(-1)^n(-1)^{i-1}\Minor_i(\hat{\vectx})&>0 & i&=1,\ldots,n-k.
\end{align}
\end{teorema}
\begin{remark}
Условие \eqref{EqualityConstraintMaxSignSufficientCondition} означает чередование знаков
в последовательности миноров $\Minor_1,\ldots,\Minor_{n-k}$, начиная со знака $(-1)^n$.
\end{remark}
	