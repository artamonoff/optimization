% !TEX root = exercise-book-opt.tex

\begin{exercise}
Решите задачи оптимизации
\begin{align*}
	&\begin{gathered}
		\min(2x^2+3y^2) \\  s.t\; x-y\geq 10
	\end{gathered} &
	&\begin{gathered}
		\max (x+2y) \\ s.t.\; x^2+y^2\leq4
	\end{gathered}
\end{align*}
\end{exercise}

\begin{exercise}
Решите задачи оптимизации
\begin{align*}
	&\begin{gathered}
		\max(2x+4y-5x^2-y^2-4xy)\\ s.t.\; 2x+3y\leq 40
	\end{gathered} &
	&\begin{gathered}
		\max(4x+8y-6x^2-2y^2-8xy)\\ s.t.\; 2x+2y\leq 9
	\end{gathered} \\
	&\begin{gathered}
		\max(4x+10y-2x^2-6y^2-6xy)\\ s.t.\; 3x+6y\leq 3
	\end{gathered} &
	&\begin{gathered}
		\max(8x+8y-3x^2-3y^2-6xy)\\ s.t.\; 2x+4y\leq 13
	\end{gathered}
\end{align*}
\end{exercise}

\begin{exercise}
Решите задачи оптимизации
\begin{align*}
	&\begin{gathered}
		\min(5x^2+y^2+4xy-2x-4y)\\ s.t.\; 2x+y\geq 40
	\end{gathered} &
	&\begin{gathered}
		\min(5x^2+y^2-4xy-2x-y)\\ s.t.\; 2x+3y\geq 10
	\end{gathered} \\
	&\begin{gathered}
		\min(5x^2+4y^2-8xy-3x-4y)\\ s.t.\; 2x+4y\geq 31
	\end{gathered} &
	&\begin{gathered}
		\min(2x^2+6y^2-6xy-4x-43)\\ s.t.\; 3x+6y\geq 20
	\end{gathered}
\end{align*}
\end{exercise}

\begin{exercise}
Решите задачи оптимизации
\begin{align*}
	& \begin{gathered}
		\max (x-2y) \\ s.t.\; x^2+y^2\leq 4
	\end{gathered} &
	& \begin{gathered}
		\max(1-(x+1)^2-(y-1)^2) \\ s.t.\; x+y\leq 10
	\end{gathered} \\
	& \begin{gathered}
		\max (x-2y) \\ s.t.\left\{\begin{aligned}
			 x^2+y^2&\leq 4 \\  x,y&\geq0
		\end{aligned}\right.
	\end{gathered} &
	& \begin{gathered}
		\max(1-(x+1)^2-(y-1)^2) \\ s.t.\left\{\begin{aligned}
			x+y&\leq 10  \\  x,y&\geq0
		\end{aligned}\right.
	\end{gathered}
\end{align*}
\end{exercise}

% \begin{exercise}
% Завод производит два вида товаров, цена на которые равны 
% $P_1=50$ и $P_2=40$. Функция издержек равна
% \[
% 	C(Q_1,Q_2)=2Q_1^2+Q_2^2
% \]
% ($Q_1, Q_2$ -- объемы производства товаров). Найдите оптимальные объемы производства,
% максимизирующие выручку, если издержки не должны
% превышать 20. 
% \end{exercise}

% \begin{exercise}%[\textbf{5 баллов} за каждый пункт]
% Завод производит два вида товаров, (обратные) функции спроса на которые
% имеют вид $P_1=50-2Q_1$ и $P_2=40-2Q_2$ (цены эндогенны). Функция
% издержек равна $C(Q_1,Q_2)=Q_1+Q_2$
% ($Q_1, Q_2$ -- объемы производства товаров).  Производитель определил,
% что издержки не должны превышать 100. Найдите оптимальную производственную программу.
% \end{exercise}

% \begin{exercise}%[\textbf{5 баллов} за каждый пункт]
% Завод производит два вида товаров, цена на которые
% равны $P_1=2$ и $P_2=4$ (цены экзогенны). Функция
% издержек равна $C(Q_1,Q_2)=Q_1^2+2Q_2^2$
% ($Q_1, Q_2$ -- объемы производства товаров). Производитель определил, что
% выручка не должна быть меньше 20.
% Найдите оптимальную производственную программу.
% \end{exercise}

% \begin{exercise}
% %\textcolor{red}{Сложная задача. }
% Потребительская корзина состоит из двух товаров, её функция
% полезности равна $U(x,y)=x+a\ln(y)$ (параметр $a>0$).
% Потребитель решает оптимальную задачу
% \begin{align*}
% 	& \max\, U(x,y) \\ 
% 	s.t.&\left\{\begin{aligned}
% 		2x+y&\leq 10 \\ x,y&\geq0
% 	\end{aligned}\right.
% \end{align*}
% При каких значениях параметрах $a$ 
% \begin{enumerate}
% 	\item потребительская корзина состоит только из 
% 	второго товара
% 	\item содержит оба товара
% \end{enumerate}
% \end{exercise}

% \begin{exercise}[Consumption--Leisure choice]
% Экономический агент имеет два <<товара>>: <<отдых>> $l$ (leisure, в часах) и потребление $x$.
% Пусть $w$ -- почасовая оплата и $P$ -- цена потребления. Агент располагает общим временем $H$,
% которое он может тратить на работу и на отдых, и также имеет фиксированный доход $M$
% (non-labor income). Функция полезности экономического агента $U(x,l)x+c\ln l$ ($c>0$). 
% Рассмотрим задачу оптимизации
% \begin{gather*}
% 	\max U(x,l) \\
% 	s.t.\left\{\begin{gathered}
% 		Px+wl\leq wH+M \\ 0\leq l\leq H \\ x\geq0
% 	\end{gathered}
% 	\right.
% \end{gather*}
% Найдите решение задачи оптимизации.
% \end{exercise}

% \begin{exercise}
% Экономический агент потребляет два товара и его функция полезности равна
% $U(x,y)=y+c\ln x$ ( $c>0$). Цены на товары равны $P_1$ и $P_2$, доход равен $I$.
% Сформулируйте задачу об оптимальной потребительской корзине и найдите её решение.
% \end{exercise}